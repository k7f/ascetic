\documentclass [a4paper,12pt] {article}
\usepackage{fullpage}
\usepackage[utf8]{inputenc}
\usepackage{amsmath, amssymb, amsthm, bm, mathtools, relsize, accents, cancel}
\usepackage{wasysym}
\usepackage{tikz, tabu}
\usepackage{stackengine}
\usepackage{enumitem}

\theoremstyle{definition}
\newtheorem{definition}{Definition}[section]
\newtheorem{lemma}{Lemma}[section]

\newcommand{\Forall}{}
\DeclareRobustCommand{\Forall}{\displaystyle\mathop{\textstyle\mathlarger{\forall}}}
\newcommand{\Exists}{}
\DeclareRobustCommand{\Exists}{\displaystyle\mathop{\textstyle\mathlarger{\exists}}}
\newcommand{\Qed}{}
\DeclareRobustCommand{\Qed}{\tag*{\qed}}
\newcommand{\inconflict}{}
\DeclareRobustCommand{\inconflict}{\mathbin{\sharp}}
\newcommand{\exconflict}{}
\DeclareRobustCommand{\exconflict}{\mathbin{\natural}}
\newcommand{\absorption}{}
\DeclareRobustCommand{\absorption}{\mathbin{\flat}}
\newcommand{\symdiff}{}
\DeclareRobustCommand{\symdiff}{\mathbin{\triangle}}

% FIXME superscript
\newcommand{\domain}{}
\DeclareRobustCommand{\domain}{\mathop{\textstyle\mathsmaller{\bf {Dom}}}}
\newcommand{\image}{}
\DeclareRobustCommand{\image}{\mathop{\textstyle\mathsmaller{\bf {Im}}}}
\newcommand{\carrier}{}
\DeclareRobustCommand{\carrier}{\mathop{\textstyle\mathsmaller{\bf {Car}}}}
\newcommand{\relative}{}
\DeclareRobustCommand{\relative}{\mathop{\textstyle\mathsmaller{\bf {Rel}}}}
\newcommand{\negative}{}
\DeclareRobustCommand{\negative}{\mathop{\textstyle\mathsmaller{\bf {Neg}}}}
\newcommand{\scope}{}
\DeclareRobustCommand{\scope}{\mathop{\textstyle\mathsmaller{\bf {Sco}}}}
\newcommand{\uniform}{}
\DeclareRobustCommand{\uniform}{\mathop{\textstyle\mathsmaller{\bf {Rep}}}}
\newcommand{\move}{}
\DeclareRobustCommand{\move}{\mathop{\textstyle\mathsmaller{\bf {Mov}}}}
\newcommand{\shift}{}
\DeclareRobustCommand{\shift}{\mathop{\textstyle\mathsmaller{\bf {Shi}}}}
\newcommand{\scaling}{}
\DeclareRobustCommand{\scaling}{\mathop{\textstyle\mathsmaller{\bf {Sca}}}}
\newcommand{\composition}{}
\DeclareRobustCommand{\composition}{\mathop{\textstyle\mathsmaller{\bf {Com}}}}

\newcommand{\id}{}
\DeclareRobustCommand{\id}{\mathop{\textstyle{\rm {id}}}}

\DeclareMathAlphabet{\xmathbb}{U}{BOONDOX-ds}{m}{n}

\newcommand{\uni}{}
\DeclareRobustCommand{\uni}{{\textstyle\mathlarger{\mathfrak{u}}}}

\newcommand{\preset}[1]{\prescript{\bullet}{}{\!#1}}
\newcommand{\postset}[1]{#1^{\bullet}}

\makeatletter
\newcommand\c@rcle[2]{\mathbin{\ooalign{\hidewidth$#1#2$\hidewidth\crcr$#1\ocircle$}}}
\newcommand\di@mond[2]{\mathbin{\ooalign{\hidewidth$#1#2$\hidewidth\crcr$#1$\raisebox{-0.06em}{$\Diamond$}}}}
\newcommand\cr@ss[2]{\mathbin{\ooalign{\hidewidth$#1#2$\hidewidth\crcr$#1$\raisebox{0.13em}{$\times$}}}}
\newcommand\bcr@ss[2]{\mathbin{\ooalign{\hidewidth$#1#2$\hidewidth\crcr$#1$\raisebox{0.09em}{$\bm{\times}$}}}}
\newcommand\bc@rc[2]{\mathbin{\ooalign{\hidewidth$#1#2$\hidewidth\crcr$#1$\raisebox{0.27em}{$\mathsmaller{\bm{\circ}}$}}}}
\newcommand{\oeq}{\mathbin{\mathpalette\c@rcle{\raisebox{0.08em}{$\mathsmaller{=}$}}}}
\newcommand{\obullet}{\mathbin{\mathpalette\c@rcle{\raisebox{0.08em}{$\mathsmaller{\bullet}$}}}}
\newcommand{\nbullet}{\mathbin{\raisebox{0.08em}{$\mathsmaller{\bullet}$}}}
\newcommand{\wxp}{\mathbin{\mathpalette\c@rcle{\raisebox{0.035em}{$\mathsmaller{\triangledown}$}}}}
\newcommand{\nxp}{\mathbin{\raisebox{0.035em}{$\mathsmaller{\triangledown}$}}}
\newcommand{\wct}{\mathbin{\mathpalette\c@rcle{\raisebox{0.085em}{$\mathsmaller{\vartriangle}$}}}}
\newcommand{\nct}{\mathbin{\raisebox{0.085em}{$\mathsmaller{\vartriangle}$}}}
\newcommand{\wtimes}{\mathbin{\mathpalette\c@rcle{\raisebox{0.085em}{$\mathsmaller{\times}$}}}}
\newcommand{\ntimes}{\mathbin{\mathpalette\di@mond{\raisebox{0.12em}{$\mathsmaller{\times}$}}}}
\makeatother

\begin {document}
\title {Firing semantics\\(a draft)}
\author {}
\date {}
\maketitle

\section {Introduction}

There is one aspect of the theory of extended cause-effect
structures\footnote{Czaja L.\ [2019], {\em Cause-Effect Structures.
    An Algebra of Nets with Examples of Applications}\/.} which is
theoretically pleasing but disturbing in practice --- compositional
properties of structural and behavioral description are deliberately
separated.  However, it is reasonable to expect a practitioner to
desire some form of behavioral transfer, which will allow to define
composite behavior based on the already defined behavior of structural
components.

Intuitively, the behavior of a system manifests itself in the way
system evolves through time in certain conditions.  In the simplest
scenario one imposes initial conditions and observes unconstrained
evolution --- a single ``run'' of a system.  In terms of cause-effect
structures, Petri nets, vector addition systems, and other similar
formalisms, single run is a sequence of {\em firings}\/, where a
single firing is understood as an atomic event quantified as a
discrete change of measurable state.

Formally, state of an extended cause-effect structure, or a
place-transition net, is represented as a function ${\mathbb
  X}\rightarrow{\mathbb N}$, where the domain ${\mathbb X}$ is any
non-empty countable set of points (variables, nodes, places, counters,
coordinates, etc.)\ and ${\mathbb N}$ is the set of non-negative
integers.  Hence, we may want to ground the structural description of
behavior in a uniform foundation consisting of functions\footnote {We
  prefer {\em not}\/ to call such functions ``signed multisets'' to
  avoid confusion with popular understanding of the term.  For
  instance, it is customary to constrain the membership in a multiset
  to elements of support, assume that operations are defined only for
  multisets sharing a common domain, etc.} from some set (or a subset
of) ${\mathbb X}$ --- called the {\em universe}\/ --- to the set (or
subset) of integers ${\mathbb Z}$.  We will assume that all these
functions share a codomain and are total over domains which are,
unless explicitly specified otherwise, unambiguously inferable subsets
of the universe ${\mathbb X}$.  The codomain may be optionally
extended with positive infinity, $\top$, to the set ${\mathbb
  Z}_\infty = {\mathbb Z}\cup\{\top\}$.

The requirements of totality and fixed codomain allow to safely
identify functions with their graphs (sets of pairs).  As usual, the
{\em restriction}\/ of a function $f \in V^X$ to some set $Y$ is the
function $f{\restriction}_Y = \{(x, v) \in f \;|\; x \in Y\}$, so that
the domains are $\domain{f} = X$ and $\domain{(f{\restriction}_Y)} = X
\cap Y$.  In general, the restriction of a relation $R \subseteq X
\times V$ to set $Y$ is the relation $R{\restriction}_Y = R \cap (Y
\times V)$, and the restriction's domain is
$\domain{(R{\restriction}_Y)} = \domain{R} \cap Y$.

We write $\bm{0}$, $\bm{1}$, $\bm{-1}$, $\bm{2}$, $\bm{-2}$ (and so on
up to $\bm{\top}$) to denote the constant functions mapping the entire
universe to respective integers (or infinity), and we accept the
existence of a single empty integer-valued function $\bm{\bot}$.  For
notational convenience, if there is no danger of confusion, we may use
the symbol $\bot$ as the placeholder for a missing value and, in the
context of a fixed $X$, identify $f{\restriction}_{Y{\subseteq}X}$
with $f{\restriction}_Y \cup \{(x, \bot) \;|\; x \in X \setminus Y\}$.

FIXME (a remark on a possible reformulation in terms of any ordered
ring)

\section {Strands}

FIXME rethink strand definition

Let $\Theta_{\mathbb X} = \{f \in {\mathbb Z}_\infty^{\mathbb Y} \;|\;
{\mathbb Y} \subseteq {\mathbb X}\}$ be the set of all integer-valued
total functions defined on subsets of the universe.  For clarity and
conciseness, the elements of $\Theta_{\mathbb X}$ will be called {\em
  strands of}\/ ${\mathbb X}$.  We also introduce the mapping
$\uni_{\mathbb X}\!: 2^{{\mathbb X} \times {\mathbb Z}_\infty}
\rightarrow \Theta_{\mathbb X}$, which assigns a strand to any
relation $r$ between points and integers in such a way, that
$\domain{(\uni_{\mathbb X} r)} = \{x \in {\mathbb X} \;|\;
|r{\restriction}_{\{x\}}| = 1\}$ and $\uni_{\mathbb X} r =
r{\restriction}_{\domain{(\uni_{\mathbb X} r)}}$ --- the result being
the {\em univalent part}\/ of $r$, i.e.\ the set $\{ (x, v) \in r
\;|\; (x, w) \in r \implies w = v\}$.  Since we may safely identify
the set $\Theta_{\mathbb X}$ with the set of fixpoints of
$\uni_{\mathbb X}$, members of the set $2^{{\mathbb X} \times {\mathbb
    Z}_\infty}$ will be called {\em prestrands of}\/ ${\mathbb X}$.

Equality of strands is interpreted in the strict sense --- as equality
of sets.  Similarly, comparison involves the entire universe: it is
the partial order $\geq$ such that $f \geq g$ iff $\domain{f} =
\domain{g}$ and $\forall_{x \in \domain{f}}\,fx \geq gx$. {\em
  Dominance}\/, on the other hand, is defined by universal
quantification over intersection of domains, i.e.\ $f \apprge g$ iff
$x \in \domain{f} \cap \domain{g} \implies fx \geq gx$.  Similarly,
{\em internal compatibility}\/ is the equality in the intersection of
domains: $f \simeq g$ iff $f{\restriction}_{\domain{g}} =
g{\restriction}_{\domain{f}}$.

Two strands having the same domain are called {\em externally
  compatible}\/.  A set of strands is (internally or externally)
compatibile iff its elements are pairwise (internally or externally)
compatible.  The definition of equality and compatibility (but not
comparison, nor dominance) extends trivially to the set of prestrands,
$2^{{\mathbb X} \times {\mathbb Z}_\infty}$.  Note, that dominance and
internal (but not external) compatibility are intransitive.

The notion of strand {\em maximality}\/ is understood as maximality of
sets: a strand $f \in \Theta_{\mathbb X}$ is maximal iff $\domain{f} =
{\mathbb X}$.  For instance, $\bm{0}$, $\bm{1}$ (and so on for any
integer $v$) are maximal constant strands.  Similarly, it may be
useful to talk about maximality of sets of strands: there are, for
example, maximal sets of externally compatible strands, internally
compatible strands, etc.

The definition of dominance and internal compatibility of strands (but
not prestrands) may be conveniently expressed in terms of operations
which we are about to introduce: $f$ dominates $g$ iff $f = (f \wxp g)
\nxp f$ and two strands are internally compatible iff their alignment
is equal to their union.

\subsection {Strand operations}

Union, intersection, complement and symmetric difference of strands
(prestrands, in general) are interpreted as set operations: $f \cup g
= \{(x, v) \;|\; (x, v) \in f \vee (x, v) \in g\}$, $f \cap g = \{(x,
v) \in f \;|\; (x, v) \in g\}$, $f \setminus g = \{(x, v) \in f \;|\;
(x, v) \not\in g\}$, $f \symdiff g = (f \cup g) \setminus (f \cap g)$,
where $f, g \subseteq {\mathbb X}\times{\mathbb Z}_\infty$.  We assume
(by accepting the axiom of choice) that union and intersection of any
countable set of prestrands are prestrands.

\begin {definition}\label{def-alignment-and-conflict}
Given a countable set $A$ of prestrands, the prestrand $\uni_{\mathbb
  X}(\bigcup A)$ is the {\em alignment}\/ of $A$, the prestrand
$\bigcup A \setminus \uni_{\mathbb X}(\bigcup A)$ is the {\em internal
  conflict}\/ of $A$, and the prestrand $\uni_{\mathbb X}(\bigcup A)
\setminus \bigcap A$ is the {\em external conflict}\/ of $A$.
\end {definition}

In words, the alignment of any number of prestrands is the univalent
part of their union, their internal conflict is the complement of
their alignment to the union, and their external conflict is the
complement of intersection to the alignment.  Note, that a set of
prestrands is (internally or externally) compatible iff its (internal
or external) conflict is empty.

For the binary case, we introduce infix notation: $f\oeq g =
\uni_{\mathbb X}(f\cup g)$, $f \inconflict g = (f \cup g) \setminus (f
\oeq g)$, and $f \exconflict g = (f \oeq g) \setminus (f \cap g)$ are
the alignment, internal conflict and external conflict of prestrands
$f$ and $g$.  Equivalently,
%
\begin {align*}
  f \inconflict g &= \{ (x, v) \in f \cup g \;|\; \Exists_{w \neq v}
  (x, w) \in f \cup g\},\\
  %
  f \exconflict g &= f{\restriction}_{{\mathbb X}\setminus
    \domain{g}}\, \cup\, g{\restriction}_{{\mathbb X}\setminus
    \domain{f}}.
\end {align*}

FIXME (revisit and assess which operations are actually useful) We
also define strand operations of: {\em absorption}\/,
%
\begin {align*}
  f \absorption g &= \{ (x, v+w) \;|\; \{(x, v), (x, w)\} \subseteq f
  \cup g \wedge vw < 0\},
\end {align*}
%
{\em wide}\/ and {\em narrow separation}\footnote{Symbol $\wtimes$
  suggests preservation of opposite (``orthogonal'') strands.}\/,
%
\begin {align*}
  f \wtimes g &= \{ (x, v) \in f \symdiff g \;|\; (x, w) \in f
  \symdiff g \implies w = v \vee vw < 0\},\\
  %
  f \ntimes g &= (f \wtimes g) \setminus (f \exconflict g),
\end {align*}
%
{\em wide}\/ and {\em narrow synchronicity}\/,
%
\begin {align*}
  f \obullet g &= \uni_{\mathbb X}(f \cup g) \cup (f \wtimes g),\\
  %
  f \nbullet g &= (f \obullet g) \setminus (f \exconflict g),
\end {align*}
%
{\em wide}\/ and {\em narrow expansion}\/,
%
\begin {align*}
  f \wxp g &= \{ (x, v) \in f \cup g \;|\; (x, w) \in f \cup g
  \implies v^2 \geq vw \geq 0\},\\
  %
  f \nxp g &= (f \wxp g) \setminus (f \exconflict g),
\end {align*}
%
and {\em wide}\/ and {\em narrow contraction}\/,
%
\begin {align*}
  f \wct g &= \{ (x, v) \in f \cup g \;|\; (x, w) \in f \cup g
  \implies v^2 \leq vw\},\\
%
f \nct g &= (f \wct g) \setminus (f \exconflict g).
\end {align*}

Other operations on strands are lifted pointwise from the addition,
subtraction and multiplication of integers and each comes in two
variants: {\em narrow}\/ (denoted $+$, $-$, $\cdot$), defined over
intersection of domains of operands, and {\em wide}\/ (denoted
$\oplus$, $\ominus$, $\odot$), defined over union of domains by
combining the narrow result with external conflict --- so for example
$f \oplus g = (f + g) \cup (f \exconflict g)$.  Negation preserves the
domain: it is a unary operation defined in terms of narrow subtraction
by the identity $-f = \bm{0} - f$.  In general, we define the
multiplication of a strand by an integer $v$ and the addition of an
integer $v$ to a strand as the narrow multiplication by, or addition
of, the constant strand $v + \bm{0} = \{(x, v) \;|\; x \in {\mathbb
  X}\}$.

\subsection {Non-algebraic properties of strand operations}

FIXME introduce notion of fusibility (compatibility with
multiplication of fcs).

Note, that set operations potentially ``fracture'' or ``poison'' the
result --- where by ``fracturing'' we mean that some elements of
intersection of operand domains are removed from domain of the result,
and by ``poisoning'' we mean that the result overlaps with internal
conflict, and thus isn't a member of $\Theta_{\mathbb X}$ --- being a
prestrand, but not a strand.  Therefore, we will also need modified
variants of set operations to ``erase'' any fractured or poisoned
result by replacing it with the empty strand, $\bm{\bot}$\footnote{One
  may want to construct a generic ``erasure'' device --- a functor
  which returns a modified operation given a strand operation and a
  predicate.}.

The notions such as ``fracturing, ``poisoning'', ``wide'', ``narrow'',
etc., represent some of the ``non-algebraic'' properties of strand
operations.  They are spelled out formally in the following
definition.

\begin {definition}\label{def-strand-operations}
  An operation $\circ\!: \Theta_{\mathbb X}\times\Theta_{\mathbb
    X}\rightarrow2^{{\mathbb X} \times {\mathbb Z}_\infty}$ --- a
  mapping from pairs of strands to prestrands ---
  %
  \begin {itemize}
  \item is {\em wide}\/ iff $\Forall_{f, g \in \Theta_{\mathbb X}} (f
    \circ g){\restriction}_{\domain{f}\symdiff\domain{g}} = f
    \exconflict g$;
    %
  \item is {\em narrow}\/ iff $\Forall_{f, g \in \Theta_{\mathbb X}}
    \domain{(f \circ g)} \subseteq (\domain{f} \cap \domain{g})$;
    %
  \item {\em fractures}\/ strand $f$ by strand $g$ in point $x \in
    {\mathbb X}$ iff $\domain{f} \cap \domain{g} \ni x \not\in
    \domain{(f \circ g)}\linebreak \neq \emptyset$;
    %
    \item {\em poisons}\/ strand $f$ by strand $g$ in point $x \in
      {\mathbb X}$ iff $f \circ g \ni (x, v) \neq (x, w) \in f \circ
      g$ for some $v, w \in {\mathbb Z}_\infty$;
    %
    \item is {\em safe}\/ iff it is total in $\Theta_{\mathbb X}$
      (which means that the set $\Theta_{\mathbb X}$ is closed under
      it) and there is no $f, g \in \Theta_{\mathbb X}$ such that
      $\circ$ fractures $f$ by $g$ in any $x \in {\mathbb X}$.
  \end {itemize}
    
  We say that an operation $\bullet\!: \Theta_{\mathbb
    X}\times\Theta_{\mathbb X}\rightarrow\Theta_{\mathbb X}$ is an
  {\em anti-fracturing erasure}\/ (resp.\ {\em anti-poisoning
    erasure}\/) of an operation $\circ\!: \Theta_{\mathbb
    X}\times\Theta_{\mathbb X}\rightarrow2^{{\mathbb X} \times
    {\mathbb Z}_\infty}$ iff given $f, g \in \Theta_{\mathbb X}$:
    %
    \begin {itemize}
      \item if there is an $x \in {\mathbb X}$ such that operation
        $\circ$ fractures (resp.\ poisons) $f$ by $g$ in $x$, then
        $f \bullet g = \bot$,
        %
      \item otherwise $f \bullet g = f \circ g$.\qed
    \end {itemize}
\end {definition}

We denote an erasure of an operator $\circ$ using the notation
$\lfloor\circ\rfloor$ subscripted with a conventional symbol of a
particular predicate: $F$ for anti-fracturing and $P$ for
anti-poisoning.  For instance, $\lfloor\!\oeq\!\rfloor_{\!{}_F}$ is
the anti-fracturing of alignment, and
$\lfloor\!\cup\!\rfloor_{\!{}_P}$ --- anti-poisoning of union.  Note,
that $f\lfloor\!\circ\!\rfloor_{\!{}_{P}}g \subseteq \uni_{\mathbb
  X}(f\circ g) \subseteq f\circ g$ for all $f, g \in \Theta_{\mathbb
  X}$ and any strand operation $\circ$.

Apart from set operations, also alignment and absorption are
fracturing FIXME list poisoning ops and safe ops.

\subsection {Algebraic properties of strand operations}

Elementary algebraic properties of specific strand operations are
described below.
%
\begin {description}
\item[Totality] The operations that aren't total in $\Theta_{\mathbb
  X}$, unless composed with $\uni_{\mathbb X}$, are: internal
  conflict, wide and narrow separation, union, and symmetric
  difference.  The remaining strand operations defined so far are
  total in $\Theta_{\mathbb X}$.
  %
\item[Associativity] Alignment, FIXME, isn't associative.  For
  example, if $fx \neq gx$ and $gx \neq hx$, then $((f \oeq g) \oeq
  h)x = hx$ but $(f \oeq (g \oeq h))x = fx$.  In general, any
  idempotent fracturing wide operation isn't associative, because if
  such an operation $\circ$ fractures $f$ by $g$ in $x$ for some point
  $x \in \domain{f} \cap \domain{g}$ and strands $f, g \in
  \Theta_{\mathbb X}$, then $((f \circ g) \circ g) x = gx \neq \bot =
  (f \circ (g \circ g)) x$.
  %
\item[Commutativity] All are commutative, except for complement and
  arithmetic difference (narrow and wide).
  %
\item[Idempotency] All are idempotent, except for external conflict,
  complement and arithmetic operations (narrow and wide).
  %
\item[Identity] Some have a neutral element:
  %
  \begin {itemize}
  \item $\bm{\bot}$ is the identity of alignment and all wide
    operations: expansion, contraction, addition, subtraction;
    %
  \item $\bm{0}$ is the identity of narrow: expansion, contraction,
    addition;
  \end {itemize}
  %
  other are not:
  %
  \begin {itemize}
  \item complement and narrow subtraction have only right identity,
    $\bm{0}$;
    %
  \item intersection and external conflict have no identity.
  \end {itemize}
  %
\item[Invertibility] FIXME Alignment admits inverses of all strands,
  however they are nowhere unique, except for $\bm{\bot}$.
  %
\item[Distributivity] FIXME
  %
\item[Derivation] FIXME
\end {description}

\[f \oeq (g \cap h) = \{(x, v) \in f \cup (g \cap h) \;|\; (x, w) \in f \cup (g \cap h) \implies w = v\}\]
%
It is easy to check that
%
\[f \cap (g \oeq h) = \{(x, v) \in f \cap (g \cup h) \;|\; (x, w) \in g \cup h \implies w = v\},\]
%
and thus if $fx = gx \neq hx$ for some $x \in \domain{f} \cap
\domain{g} \cap \domain{h}$, then $x \not\in \domain{(f \cap (g \oeq
  h))}$, but $x \in \domain{((f \cap g) \oeq (f \cap h))}$.

FIXME duality of narrow and wide operations, e.g.\ $\bm{0}$ and
$\bm{\bot}$ are either neutral or anihilating under addition, etc.

\section {Transitions}

We now introduce two equivalent descriptions of context-free state
transitions.  The transitions are ``context-free'' in the sense that
the variables are modified independently of each other (they are also
``memory-less'' in the sense that the next state is a function of the
current state only).  As usual, we assume that a universe of
variables, a non-empty countable set ${\mathbb X}$, is given.

\begin {definition}\label{def-free-move}
  A {\em free move over}\/ ${\mathbb X}$ is any function from the set
  of integers to a set of externally compatible strands over ${\mathbb
    X}$.  The {\em carrier}\/ of a free move is the common domain of
  all its strands.

  Given a free move $\mu$ and a point $x$ in the carrier of $\mu$, let
  $\mu_x$ denote a function ${\mathbb Z} \rightarrow {\mathbb Z}$,
  called a {\em projection}\/, such that $\mu_x v = (\mu v) x$ for any
  $v \in {\mathbb Z}$.  We say that $\mu$ is
  %
  \begin {itemize}
  \item {\em reversible}\/ iff $\mu_x$ is injective, i.e.\ $\mu_x v =
    \mu_x w \implies v = w$, in particular that it is
    %
  \item {\em affine}\/ iff $\mu_x v - \mu_x w \propto v - w$, and in
    particular that it is
    %
  \item {\em rigid}\/ iff $\mu_x v - \mu_x w = v - w$,
  \end {itemize}
  %
  for any two integers $v$ and $w$, and any element $x$ of the
  carrier.  If a free move $\mu$ is affine, then the strand $\mu 0$ is
  called its {\em shift}\/, and the strand $\mu 1 - \mu 0$ is called
  its {\em scaling}\/.

  The {\em bottom free move}\/, ${\xmathbb 0}$, maps all integers to
  the empty strand.  The {\em null free move}\/, ${\xmathbb 1}$, maps
  any integer $v$ to the maximal constant strand $v + \bm{0}$.\qed
\end {definition}

We also define two unary operations,
%
\begin {align*}
  (\relative{\mu})v &= \mu v - v,\\
  (\negative{\mu})v &= \mu (-v),
\end {align*}
%
where $\mu$ is any free move and $v$ is any integer. $\relative$ and
$\negative$ are bijections in the set of all free moves.
Specifically, $\negative^{-1} = \negative = \relative . \negative
. \relative$, and $\relative^{-1} = \negative . \relative
. \negative$.

All safe strand operations directly translate to {\em absolute}\/ free
move operations by applying an operation separately to all
corresponding strands of the operand moves.  With any absolute free
move operation there is associated the {\em relative}\/ counterpart.
For instance, given an absolute binary free move operation $\bullet$,
the corresponding relative free move operation $\circ$ is defined so
that for any two free moves $\mu$ and $\nu$,
%
\[\mu \circ \nu = {\relative}^{-1}{(\relative{\mu} \bullet
  \relative{\nu})}.\]

We may picture a free move as an integer matrix, which rows are
indexed by the set of integers and columns are indexed by the carrier.
Column vectors are projections, row vectors are strands.  A free move
induces state transition deterministically up to the carrier: if the
current value of a variable $x$ is $v$, then (if $x$ is in the
carrier) we read the next value from the matrix entry in column $x$
and row $v$.

Evolution of state may also be captured, perhaps more explicitly, by
the equivalent notion of a ``free shot''.

\begin {definition}\label{def-free-shot}
  A map $\tau\!: \Theta_{\mathbb X} \rightarrow \Theta_{\mathbb X}$ is
  a {\em free shot over}\/ ${\mathbb X}$ iff there is a set ${\mathbb
    Y} \subseteq {\mathbb X}$, called the {\em carrier}\/ of $\tau$,
  such that
  %
  \begin {enumerate}
  \item $\domain{(\tau f)} = \domain{f} \cap {\mathbb Y}$ for any $f
    \in \Theta_{\mathbb X}$, and
    %
  \item $fx = gx$ implies $(\tau f)x = (\tau g)x$ for any $f, g \in
    \Theta_{\mathbb X}$ and $x \in \domain{f} \cap \domain{g} \cap
          {\mathbb Y}$.
  \end {enumerate}

  Given a free shot $\tau$ and a point $x$ in the carrier of $\tau$,
  let $\tau_x$ denote a function ${\mathbb Z} \rightarrow {\mathbb
    Z}$, called a {\em projection}\/, such that $\tau_x v = (\tau (v +
  \bm{0})) x$ for any $v \in {\mathbb Z}$.  We say that $\tau$ is
  %
  \begin {itemize}
  \item {\em reversible}\/ iff $\tau_x$ is injective, i.e.\ $\tau_x v
    = \tau_x w \implies v = w$, in particular that it is
    %
  \item {\em affine}\/ iff $\tau_x v - \tau_x w \propto v - w$, and in
    particular that it is
    %
  \item {\em rigid}\/ iff $\tau_x v - \tau_x w = v - w$,
  \end {itemize}
  %
  for any two integers $v$ and $w$, and any element $x$ of the
  carrier.  If a free shot $\tau$ is affine, then the strand
  $\tau\bm{0}$ is called its {\em shift}\/, and the strand $\tau\bm{1}
  - \tau\bm{0}$ is called its {\em scaling}\/.  The {\em bottom free
    shot}\/ is the constant mapping of all strands to the empty
  strand.\qed
\end {definition}

Obviously, there is a projection-preserving isomorphism between the
set of all free shots with the carrier ${\mathbb Y}$ and the set of
all free moves ${\mathbb Z} \rightarrow {\mathbb Z}^{\mathbb Y}$.
Given a free move $\mu$, the corresponding free shot is written as
$[\mu\rangle$.  For convenience, the bottom and identity free shots
  are denoted by the symbols ${\xmathbb 0}$ and ${\xmathbb 1}$,
  respectively, instead of $[{\xmathbb 0}\rangle$ and $[{\xmathbb
        1}\rangle$.

It is easy to check that a free shot $\tau$ is affine iff $\tau f =
(\tau\bm{1} - \tau\bm{0}) \cdot f + \tau\bm{0}$, and it is rigid iff
$\tau f = f + \tau\bm{0}$, for any strand $f$.  From the last identity
we can derive $\tau_x (\tau_x v) = \tau_x (v + \tau_x 0) = v + 2\tau_x
0$, and similarly, $\tau_x^{-1} (\tau_x^{-1} v) = \tau_x^{-1} (v -
\tau_x 0) = v - 2\tau_x 0$, for any $x$ in the carrier and any integer
$v$.  In general, $\tau_x^k v = v + k\tau_x 0$, and after setting
$\tau^0 = {\xmathbb 1}$ we may express the evolution of a maximal
constant strand by the formula
%
\[\tau^k f = f + k\tau\bm{0}\]
%
where $\tau$ is any rigid free shot and $k, v$ are any integers.

Another easily derived property of rigid free shots is
``transadditivity''.  If $f$ and $g$ are externally compatible
strands, then for any rigid $\tau$, FIXME (replace subtraction with
scaling)
%
\begin {align*}
  \tau (f + g) &= \tau f + \tau g - \tau\bm{0}\textrm {, and}\\
  %
  \tau (f - g) &= \tau f - \tau g + \tau\bm{0}.
\end {align*}

Next, we introduce a globally constrained move, enabled in some
states, but disabled in others --- a simple form of
context-dependence.

\begin {definition}\label{def-motion}
  A {\em motion over}\/ ${\mathbb X}$ is any set ${\cal M}$, together
  with three functions, $\carrier$, $\scope$ and $\move$, of which
  ${\cal M}$ is the domain, where for any $a \in {\cal M}$
  %
  \begin {itemize}
  \item $\carrier{a} \subseteq {\mathbb X}$ is a subset of the
    universe called the {\em carrier}\/ of $a$,
    %
  \item $\scope{a} \subseteq \Theta_{\mathbb X}$ is a set of strands
    called the {\em scope}\/ of $a$, and
    %
  \item $\move{a}\!: {\mathbb Z} \rightarrow {\mathbb
    Z}^{\carrier{a}}$ is a free move,
  \end {itemize}
  %
  and such that ${\cal M}$ is closed under narrow addition, defined as
  $\carrier{(a + b)} = \carrier{a} \cap \carrier{b}$, $\scope{(a + b)}
  = \scope{a} \cap \scope{b}$, $\move{(a + b)} = \move{a} + \move{b}$,
  and under multiplication by an integer constant, defined as
  $\carrier{(va)} = \carrier{a}$, $\scope{(va)} = \scope{a}$,
  $\move{(va)} = v\move{a}$.  We say that ${\cal M}$ is {\em
    formed}\/ by the functions $\carrier$, $\scope$ and $\move$.

  Elements of a motion are called {\em steps}\/.  A step $a$ is
  reversible (resp.\ affine, resp.\ rigid) iff $\move{a}$ is
  reversible (resp.\ affine, resp.\ rigid).  If $a$ is affine, then
  its {\em shift}\/ is denoted $\shift{a} = (\move{a}) 0$ and its {\em
    scaling}\/ is denoted $\scaling{a} = (\move{a}) 1 - (\move{a}) 0$.

  The {\em uniform representation}\/ of a step $a \in {\cal M}$ is the
  tripple $\uniform{a} = (\carrier{a}, \scope{a}, \move{a})$.  A {\em
    bottom step}\/ is any step whose uniform representation is
  $(\emptyset, \emptyset, {\xmathbb 0})$ and a {\em null step}\/ is a
  step whose uniform representation is $({\mathbb X}, {\mathbb
    Z}^{\mathbb X}, {\xmathbb 1})$.

  Given any step $a$, the {\em shot}\/ of $a$ is the mapping
  $[a\rangle\!: \scope{a} \rightarrow {\mathbb Z}^{\carrier{a}}$, such
    that $({[a\rangle}s)x = (\move{a})_x(sx)$ for all $x \in
      \carrier{a}$ and $s \in \scope{a}$.  Given a set of steps ${\cal
        N} \subseteq {\cal M}$, the corresponding set of shots is
      $[{\cal N}\rangle = \{[a\rangle \;|\; a \in {\cal N}\}$.

  An {\em antishot}\/ of $a \in {\cal M}$ is any mapping
  $f\!:\image{[a\rangle} \rightarrow {\mathbb Z}^{\carrier{a}}$
    satisfying $f({[a\rangle}s) = s{\restriction}_{\carrier{a}}$ for
      all $s \in \scope{a}$.\qed
\end {definition}

FIXME (should this be an axiom? do we want it) Given any motion ${\cal
  M}$, there is a bijection between ${\cal M}$ and the set
$\uniform{{\cal M}} = \{\uniform{a} \;|\; a \in {\cal M}\}$.
Therefore, $\uniform{{\cal M}}$ is also a motion FIXME formed by
$\uniform^{-1}\carrier{{\cal M}}$, $\scope$ and $\move$ of ${\cal M}$.
FIXME at most one bottom and one null step.

The shot of the bottom step is empty, and so is the antishot.  The
equality of shot and antishot is also true for the null step, and in
general, $[({\mathbb X}, S, {\xmathbb 1})\rangle = \langle({\mathbb
    X}, S, {\xmathbb 1})] = \id_S$ for any $S \subseteq {\mathbb
  Z}^{\mathbb X}$.

If step $a$ is reversible, then it has exactly one antishot, denoted
$\langle a]$.  Let, for example, $\uniform{a} = (\{x\}, \{\{(x, 1)\},
    \{(x, 2)\}\}, \mu)$.  If $\mu 1 = \{(x, 2)\}$ and $\mu 2 = \{(x,
    1)\}$, then $\langle a] = \{(\{(x, 1)\}, \{(x, 2)\}), (\{(x, 2)\},
  \{(x, 1)\})\}$, but if $\mu 1 = \mu 2$, then $a$ has no antishot.

FIXME $\langle a](\move{a})_x(sx) = sx$

For a rigid step $a$, maximal strands $s_1, s_2 \in \scope{a}$, and $x
\in \carrier{a}$, we obtain from transadditivity property,
%
\begin {align*}
  {[a\rangle}(s_1 + s_2) &= {[a\rangle}s_1 +
      {[a\rangle}s_2 - \shift{a}\textrm{, and}\\
        %
  {[a\rangle}(s_1 - s_2) &= {[a\rangle}s_1 -
      {[a\rangle}s_2 + \shift{a}.
\end {align*}

On the other hand, $({[a\rangle}s_1)x - ({[a\rangle}s_2)x =
    ((\move{a})(s_1x))x - ((\move{a})(s_2x))x = s_1x - s_2x$, and thus
    ${[a\rangle}s_1 - {[a\rangle}s_2 = (s_1 -
        s_2){\restriction}_{\carrier{a}}$.  In particular, since
        ${[a\rangle}\bm{0} = \shift{a}$, we have ${[a\rangle}s -
            \shift{a} = (s - \bm{0}){\restriction}_{\carrier{a}} =
            s{\restriction}_{\carrier{a}}$, and finally
%
\[\shift{a} = {[a\rangle}s - s.\]
%
for any rigid step $a$ and maximal strand $s$.  We may now rewrite
transadditivity formulae as
%
\begin {align*}
  {[a\rangle}(s_1 + s_2) &= {[a\rangle}s_1 +
      s_2 = s_1 + {[a\rangle}s_2\textrm{, and}\\
        %
  {[a\rangle}(s_1 - s_2) &= {[a\rangle}s_1 - s_2,
\end {align*}
%
or --- after deriving ${\langle a]}(s + \shift{a}) =
    s{\restriction}_{\carrier{a}}$ for antishot --- as FIXME recheck
%
\begin {align*}
  {[a\rangle}(s_1 + s_2) &= {\langle a]}({[a\rangle}s_1
    + {[a\rangle}s_2)\textrm{, and}\\
      %
  {[a\rangle}(s_1 - s_2) &= {\langle a]}({[a\rangle}s_1
    - {[a\rangle}s_2).
\end {align*}

FIXME introduce delta.

\begin {definition}\label{def-step}
  A scope that is a hyperinterval (a box) is called a {\em slice}\/.

  A {\em sliced rigid move over}\/ ${\mathbb X}$ is any triple of
  strands $(\beta, \gamma, \delta) \in {\mathbb Z}^{\mathbb
    Y}\!\times\!{\mathbb Z}_\infty^{\mathbb Y}\!\times\!{\mathbb
    Z}^{\mathbb Y}$, such that ${\mathbb Y} \subseteq {\mathbb X}$ and
  $\beta \leq \gamma$.  The common domain of component strands,
  i.e.\ the set $\carrier{\sigma} = {\mathbb Y} = \domain{\beta} =
  \domain{\gamma} = \domain{\delta}$, is called the {\em carrier}\/ of
  a sliced rigid move $\sigma = (\beta, \gamma, \delta)$, and the set
  of maximal strands $\scope{\sigma} = \{s \in {\mathbb Z}^{\mathbb X}
  \;|\; \beta \apprle s \apprle \gamma\}$ is called the {\em scope
    of}\/ $\sigma$ (we say that a strand $s$ is {\em in scope of}\/
  $\sigma$ iff $s \in \scope{\sigma}$).

  The {\em shot}\/ of $\sigma$, where $\sigma = (\beta, \gamma,
  \delta)$ is any sliced rigid move over ${\mathbb X}$, is the mapping
  $[\sigma\rangle\!: \scope{\sigma} \rightarrow {\mathbb Z}^{\mathbb
      X}$, such that ${[\sigma\rangle}s = s \oplus \delta$.

  A {\em firing step over}\/ ${\mathbb X}$, or just {\em step}\/, is a
  sliced rigid move over ${\mathbb X}$ for which $\beta \geq -\delta$.
  The set of all non-bottom firing steps over ${\mathbb X}$ is denoted
  $\Sigma_{\mathbb X}$.
\end {definition}

FIXME The triple $(\bm{\bot}, \bm{\bot}, \bm{\bot})$ is the bottom
step.  The set of all steps is $\Sigma_{\mathbb X} \cup \{(\bm{\bot},
\bm{\bot}, \bm{\bot})\}$.

For instance, the triple $(\bm{0}, \bm{0}, \bm{0})$ is a step and so
is $(\bm{1}, \bm{1}, \bm{1})$, $(\bm{0}, \bm{0}, \bm{1})$, and
$(\bm{1}, \bm{1}, \bm{-1})$, but none of $(\bm{\bot}, \bm{0},
\bm{0})$, $(\bm{1}, \bm{0}, \bm{1})$ and $(\bm{0}, \bm{1}, \bm{-1})$
is --- the last one being a sliced rigid move.  The shot of the bottom
step is empty, $[(\bm{0}, \bm{0}, \bm{0})\rangle = \{(\bm{0},
  \bm{0})\}$, $[(\bm{0}, \bm{1}, \bm{1})\rangle = \{(\bm{0}, \bm{1}),
    (\bm{1}, \bm{2})\}$, $[(\bm{1}, \bm{1}, \bm{-1})\rangle =
      \{(\bm{1}, \bm{0})\}$, $[(\bm{0}, \bm{\top}, \bm{0})\rangle =
        \{(\bm{0}, \bm{0}), (\bm{1}, \bm{1}), (\bm{2}, \bm{2}),
        \ldots\}$, etc.

Note, that if $\sigma \in \Sigma_{\mathbb X}$ and $s \in
\scope{\sigma}$, then $\bm{0} \leq [{\sigma\rangle}s \leq \bm{0}
  \oplus (\gamma + \delta)$.  Note also, that $\Sigma_{\mathbb X}$ is
  infinite in any universe: $|\Sigma_{\mathbb X}| = |{\mathbb N}|$ if
  ${\mathbb X}$ is finite and $|\Sigma_{\mathbb X}| = |{\mathbb R}|$
  otherwise.

FIXME a function ${\mathbb Z} \times \carrier{\sigma} \rightarrow
{\mathbb Z}$, such that FIXME $\delta_\sigma \in {\mathbb Z}^{{\mathbb
    Z} \times \carrier{\sigma}}$.

Note that all rigid shots are injective (${\mathbb Z}^{\mathbb X}$
forms a group under $\oplus$).  Therefore, we may introduce the {\em
  antishot}\/ of $\sigma$, $\langle\sigma]$

FIXME analyze $\Sigma_{\mathbb X}$ as a poset under scope inclusion.

FIXME translation to and from VAS.

\begin {definition}\label{def-step-operation}
  We declare, that if any relation $R$ is defined for strands, then it
  is also defined for steps,
  %
  \[R((\beta_1, \gamma_1, \delta_1), (\beta_2, \gamma_2, \delta_2)) \iff
  R(\beta_1, \beta_2) \wedge R(\gamma_1, \gamma_2) \wedge R(\delta_1, \delta_2).\]
  %
  Any such $R$ is called a~{\em step relation}\/.  Moreover, if
  $\circ$ is an operation total in $\Theta_{\mathbb X}$, then it
  extends to a {\em step operation}\/ in the following way:
  %
  \[(\beta_1, \gamma_1, \delta_1) \circ (\beta_2, \gamma_2, \delta_2) =
  ((\beta_1 \circ \beta_2){\restriction}_{\carrier}, (\gamma_1 \circ
  \gamma_2){\restriction}_{\carrier}, (\delta_1 \circ
  \delta_2){\restriction}_{\carrier}),\]
  %
  where $\carrier = \domain{(\beta_1 \circ \beta_2)} \cap
  \domain{(\gamma_1 \circ \gamma_2)} \cap \domain{(\delta_1 \circ
    \delta_2)}$ is the carrier of the result.
\end {definition}

FIXME $(\beta_1, \gamma_1, \delta_1) \obullet (\beta_2, \gamma_2,
\delta_2) = ((\beta_1 \wct \beta_2){\restriction}_{\carrier},
(\gamma_1 \wxp \gamma_2){\restriction}_{\carrier}, (\delta_1 \oeq
\delta_2){\restriction}_{\carrier})$

Union and internal conflict are not step operations.

FIXME algebraic structure of alignment, conflict, expansion of steps.

FIXME dominance and compatibility.

\begin {definition}\label{def-layers}
  For any step $\sigma = (\beta, \gamma, \delta) \in \Sigma_{\mathbb
    X}$ we define a partition of the universe into up to six disjoint
  subsets,
  %
\begin {align*}
  {\cal A}(\sigma) &\;=\; \{x \in \carrier{\sigma} \;|\; \beta x > 0
  \wedge \delta x = 0\},\\
  %
  {\cal B}(\sigma) &\;=\; \{x \in \carrier{\sigma} \;|\; \beta x =
  \gamma x = \delta x = 0\},\\
  %
  {\cal C}(\sigma) &\;=\; \{x \in \carrier{\sigma} \;|\; \delta x <
  0\},\\
  %
  {\cal D}(\sigma) &\;=\; \{x \in \carrier{\sigma} \;|\; \beta x =
  \delta x = 0 \wedge \gamma x > 0\},\\
  %
  {\cal E}(\sigma) &\;=\; \{x \in \carrier{\sigma} \;|\; \delta x >
  0\},\\
  %
  {\cal F}(\sigma) &\;=\; {\mathbb X} \setminus \carrier{\sigma},
\end {align*}
%
called, in order: {\em activators}\/, {\em inhibitors}\/, {\em
  sources}\/, {\em dummies}\/, {\em sinks}\/ and {\em residue}\/.  By
applying appropriate restrictions to component strands of $\sigma$
we get {\em activating layer}\/ $\sigma{\restriction}_{\cal A}\;=\;
(\beta{\restriction}_{\cal A}, \gamma{\restriction}_{\cal A},
\delta{\restriction}_{\cal A})$, and similarly, {\em blocking}\/
($\sigma{\restriction}_{\cal B}$), {\em contracting}\/
($\sigma{\restriction}_{\cal C}$), {\em dummy}\/
($\sigma{\restriction}_{\cal D}$) and {\em expanding}\/
($\sigma{\restriction}_{\cal E}$) {\em layers of}\/ $\sigma$.\qed
\end {definition}

Two important examples of strands are the capacity and weight
functions of extended c-e structures.  Since capacity is defined for a
specific structure $U$, the domain of a capacity strand $\widehat{U}$
is the carrier of $U$.  Similarly, the domain of a weight strand
$\widetilde{Q}_U$, where $U$ is some structure and $Q$ is one if its
firing components, is the carrier of $Q$: the set $\preset{Q} \cup
\postset{Q}$.  If we apply definition~\ref{def-layers} to extended c-e
structures, then ${\cal A}$ will be the set of activators (zero-weight
members of the pre-set), ${\cal B}$ --- the set of inhibitors
($\omega$-weight members of the pre-set), ${\cal C} = \preset{Q}
\setminus ({\cal A}\cup{\cal B})$ will be the set of sources, ${\cal
  D}$ --- the set of dummies (zero-weight members of the post-set),
${\cal E} = \postset{Q}\setminus {\cal D}$ will be the set of sinks
(positive-weight members of the post-set), and ${\cal F}={\mathbb
  X}\setminus({\cal A}\cup{\cal B}\cup{\cal C}\cup{\cal D}\cup{\cal
  E})$ --- the set of isolated nodes.

\begin {definition}\label{def-step-of-fc}
  Let $Q$ be a firing component of a c-e structure $U$ over ${\mathbb
    X}$.  The {\em firing step of}\/ $Q$ {\em in}\/ $U$ is the natural
  step $(\beta_Q, \gamma_Q, \delta_Q)$ such that
  %
\begin {align*}
  \beta_Q &\;=\; \bm{0}{\restriction}_{{\cal B} \cup {\cal E}}
  \;\cup\; \bm{1}{\restriction}_{\cal A} \;\cup\;
  \widetilde{Q}_U{\restriction}_{\cal C},\\
  %
  \gamma_Q &\;=\; \bm{0}{\restriction}_{\cal B} \;\cup\;
  \widehat{U}{\restriction}_{{\cal A} \cup {\cal C}} \;\cup\;
  (\widehat{U} \ominus \widetilde{Q}_U){\restriction}_{\cal E},\\
  %
  \delta_Q &\;=\; \bm{0}{\restriction}_{{\cal A} \cup {\cal B}} \;\cup\;
  \widetilde{Q}_U{\restriction}_{\cal E} \;\cup\;
  -\widetilde{Q}_U{\restriction}_{\cal C}.\Qed
\end {align*}
\end {definition}

FIXME reverse mapping from steps to capacity, weight and polarity.
Rules for delta zero disambiguation: inhibitor, activator, or
anihilator?

\section {Causality}

Behavioral description of systems in terms of moves, steps and shots,
however useful for simulation, doesn't sufficiently capture all
compositional aspects.  For an adequate rendering of the combined
behavior of multiple systems we need a detailed knowledge about causal
structure of individual components.

\begin {definition}\label{def-causal-component}
  A {\em move-labeled}\/ (resp.\ {\em step-labeled}\/) {\em firing
    graph over}\/ ${\mathbb X}$ is a pair $(\nu, \eta)$
  (resp.\ $(\sigma, \eta)$), where $\nu$ is a move (resp.\ $\sigma$ is
  a step) over ${\mathbb X}$ and $\eta$ is an edge set of a simple
  undirected graph $G$, such that $\carrier{\nu}$
  (resp.\ $\carrier{\sigma}$) is exactly the vertex set of $G$.  The
  pair $({\xmathbb 0}, \emptyset)$ is called the {\em bottom graph}\/.
  A firing graph is {\em causal}\/ iff $G$ is bipartite, and a
  non-bottom causal graph is called a {\em causal component}\/ iff $G$
  is connected.  $\Gamma(\Omega_{\mathbb X})$ and
  $\Gamma(\Sigma_{\mathbb X})$ are the sets of all move-labeled and
  all step-labeled causal components over ${\mathbb X}$.

  The shot of a firing graph is the shot of its move or step.  The
  shot notation is extended to firing graphs: $[(\nu, \eta)\rangle =
    [\nu\rangle$ and $[(\sigma, \eta)\rangle = [\sigma\rangle$.

  FIXME causal operations\qed
\end {definition}

\begin {definition}\label{def-grill}
  A {\em grill over}\/ ${\mathbb X}$ is a finite set of causal
  components --- a finite subset of $\Gamma(\Omega_{\mathbb X})$ or
  $\Gamma(\Sigma_{\mathbb X})$.  The set of all grills is
  $\bm{\Omega}_{\mathbb X} = \bigcup_{n\in{\mathbb
      N}}\binom{\Gamma(\Omega_{\mathbb X})}{n}$ --- move-labeled ---
  or $\bm{\Sigma}_{\mathbb X} = \bigcup_{n\in{\mathbb
      N}}\binom{\Gamma(\Sigma_{\mathbb X})}{n}$ --- step-labeled.

  FIXME moveset of $\bm{a}$.

  A {\em grill product}\/ is any operation $\bm{\circ}\!:
  \bm{\Omega}_{\mathbb X} \times \bm{\Omega}_{\mathbb X} \rightarrow
  \bm{\Omega}_{\mathbb X}$ for which there is a causal operation
  $\circ$, such that FIXME $\bm{a} \bm{\circ} \bm{b} = (\bigcup_{(g,h)
    \in \bm{a} \times \bm{b}}g\circ h) \setminus \{({\xmathbb 0},
  \emptyset)\}$ for all $\bm{a}, \bm{b} \in \bm{\Omega}_{\mathbb
    X}$.\qed
\end {definition}

FIXME infinity class of $\bm{\Omega}_{\mathbb X}$.

We denote a grill product operator by the same symbol (if there is
any) as the corresponding step operator, but typeset in bold face.
Hence there are, for example: {\em grill $\bm{\obullet}$-product}\/,
{\em grill $\bm{\oeq}$-product}\/, etc.

\begin {definition}\label{def-shell}
  Let $\bm{a} \in \bm{\Omega}_{\mathbb X}$ (or $\bm{a} \in
  \bm{\Sigma}_{\mathbb X}$) be a grill over ${\mathbb X}$.  The
  $\cup$\,-{\em shell}\/, or {\em union-shell}\/ in words
  (resp.\ $\cap$\,-{\em shell}\/ or {\em intersection-shell}\/) of
  $\bm{a}$ is the union (resp.\ intersection) of all shots of causal
  components in $\bm{a}$:
  %
  \begin {align*}
    [\bm{a}\rangle_\cup &= \bigcup_{a \in \bm{a}}[a\rangle = \{(s, t)
        \in \Theta_{\mathbb X}\!\times\!\Theta_{\mathbb X} \;|\;
        \Exists_{a \in \bm{a}}[a\rangle s = t\},\\
    %
    [\bm{a}\rangle_\cap &= \bigcap_{a \in \bm{a}}[a\rangle = \{(s, t)
        \in \Theta_{\mathbb X}\!\times\!\Theta_{\mathbb X} \;|\;
        \Forall_{a \in \bm{a}}[a\rangle s = t\}.
  \end {align*}

  A strand relation $\bm{s} \subseteq \Theta_{\mathbb X} \times
  \Theta_{\mathbb X}$ is called a {\em firing shell over}\/ ${\mathbb
    X}$ iff it is the union-shell of some grill $\bm{a}$ over
  ${\mathbb X}$: $\bm{s} = [\bm{a}\rangle_\cup$.\qed
\end {definition}

FIXME clans of union-shells.

\section {Expressions}

FIXME homomorphism problem: does sum/product of any two extended
polynomials reduce to the sum/product of reduced component
polynomials?  Is any extended polynomial constructible/normalizable?

The set $\mathfrak{R}_{\mathbb X}$ of firing expressions over
${\mathbb X}$ is defined recursively along with the labelset function
$\lambda$, mapping firing expressions to subsets of ${\mathbb Z}_+$,
and the composition function $\composition$, mapping firing
expressions to sets of firing expressions.

\begin {definition}\label{def-expression}
  A {\em firing expression} $\bm{r}$ over ${\mathbb X}$, its {\em
    labelset}\/ $\lambda\bm{r}$ and {\em composition}\/
  $\composition\bm{r}$ are one of the following:
  %
  \begin {enumerate}
  \item any grill, with empty labelset and empty composition,
    i.e.\ $\lambda{\bm{r}} = \emptyset$ and $\composition{\bm{r}} =
    \emptyset$ for any $\bm{r} \in \bm{\Sigma}_{\mathbb X} \subset
    \mathfrak{R}_{\mathbb X}$;
    %
  \item any triple $\bm{r} = (\bm{p}, \bm{q}, k)$, with labelset
    $\lambda{\bm{r}} = \lambda\bm{p} \cup \lambda\bm{q} \cup \{k\}$,
    and composition $\composition{\bm{r}} = \composition\bm{p} \cup
    \composition\bm{q} \cup \{\bm{r}\}$, where $k$ is a positive
    integer and $\bm{p}$ and $\bm{q}$ are firing expressions over
    ${\mathbb X}$ --- not necessarily different, but such that
    %
    \begin {enumerate}
    \item neither is an empty grill, and
      %
    \item $\Forall_{\bm{o} \in
      \composition\bm{p}\,\cup\,\composition\bm{q}} \bm{p} \in
      \composition\bm{o} \iff \bm{p} = \bm{o} \,\wedge\, \bm{q} \in
      \composition\bm{o} \iff \bm{q} = \bm{o}$.
    \end {enumerate}
  \end {enumerate}
  %
  There are no more firing expressions over ${\mathbb X}$.  The {\em
    extent}\/ of a firing expression $\bm{r}$ is the number
  $|\lambda\bm{r}|$.\qed
\end {definition}

The relation ``is composed of'' defined below, a partial order in the
set of firing expressions, follows in a natural way from the
composition function.  Note, however, that firing expressions aren't
composed of grills --- by definition, composition contains only
``composite'' expressions.  Therefore, we also introduce a slightly
broader notion of ``subexpression''.

\begin {definition}\label{def-subexpression}
  Given two firing expressions $\bm{r}, \bm{q} \in
  \mathfrak{R}_{\mathbb X}$, we say that $\bm{r}$ {\em is composed
    of}\/ $\bm{q}$ iff $\bm{q} \in \composition{\bm{r}}$, and we say
  that $\bm{q}$ is a {\em proper subexpression}\/ of $\bm{r}$ iff
  $\bm{r}$ is composed of a firing expression $(\bm{q}, \bm{p}, k)$ or
  $(\bm{p}, \bm{q}, k)$ for some firing expression $\bm{p}$ and
  integer $k$.  A {\em subexpression}\/ of $\bm{r}$ is any proper
  subexpression of $\bm{r}$, or the $\bm{r}$ itself.\qed
\end {definition}

The set $\mathfrak{R}_{\mathbb X}$ may be described as a set of nodes
of a directed graph, where each node has an uncountable set of
incoming edges, and either two outgoing edges, or none.  There is a
directed path from node $\bm{r}$ to node $\bm{q}$ iff firing
expression $\bm{r}$ has a subexpression $\bm{q}$.  The directed graph
is acyclic: condition 2b.\ of definition~\ref{def-expression}
guarantees that no firing expression is a proper subexpression of
itself.

Obviously, the underlying undirected graph isn't acyclic: every
expression is a subexpression in multiple ``instances'' (more
formally, for every expression there is another expression, such that
the two are connected via multiple paths).  However, the structure of
$\mathfrak{R}_{\mathbb X}$ may be ``unfolded'' into a tree by
assigning nodes to individual instances.

Similarly, a subgraph induced by a single firing expression and all
its subexpressions may be unfoled into a tree.  Any such tree is
finite and binary, with internal nodes labeled by positive integers
and with non-empty grills in leaves.

FIXME the need for syntactic equivalence, root expression vs
subexpressions.

\begin {definition}\label{def-e-expression}
  A {\em canonical labelset}\/ is an empty set or any interval $[1,
    e]$, where $e \in {\mathbb Z}_+$.  A firing expression $\bm{q}$ is
  {\em canonical}\/ iff $\lambda\bm{q}$ is a canonical labelset.  A
  canonical firing expression of extent $e$ is called an
  $e$-expression.  The set of all $e$-expressions over ${\mathbb X}$
  is denoted $\mathfrak{R}_{\mathbb X}^e \subset \mathfrak{R}_{\mathbb
    X}$.\qed
\end {definition}

Note, that a canonical firing expression may have non-canonical
subexpressions.

\begin {definition}\label{def-relabeling}
  Given a firing expression $\bm{r} \in \mathfrak{R}_{\mathbb X}$ and
  a function $f\!: \lambda\bm{r} \rightarrow {\mathbb Z_+}$, the {\em
    $f$-relabeling}\/ of $\bm{r}$ is the firing expression $\bm{r}^f
  \in \mathfrak{R}_{\mathbb X}$ such that
  %
  \begin {enumerate}
    \item if $\bm{r} \in \bm{\Sigma}_{\mathbb X}$, then $\bm{r}^f =
      \bm{r}$;
      %
    \item if $\bm{r} = (\bm{p}, \bm{q}, k)$, then $\bm{r}^f =
      (\bm{p}^f, \bm{q}^f, fk)$.
  \end {enumerate}

  Two firing expressions $\bm{r}, \bm{q} \in \mathfrak{R}_{\mathbb X}$
  are {\em isomorphic}\/ (or {\em syntactically equivalent}\/),
  symbolically $\bm{r} \cong \bm{q}$, iff there is an order preserving
  bijection $f\!: \lambda\bm{r} \rightarrow \lambda\bm{q}$, such that
  $\bm{q}$ is the $f$-relabeling of $\bm{r}$.\qed
\end {definition}

Note, that for any $\bm{r} \in \mathfrak{R}_{\mathbb X}$ there is
exactly one isomorphic canonical firing expression $\bm{q} \in
\mathfrak{R}_{\mathbb X}^{|\lambda\bm{r}|}$.  FIXME explain (exactly
one order preserving bijection).

\begin {definition}\label{def-intent}
  A canonical firing expression $\bm{q} \in \mathfrak{R}_{\mathbb
    X}^{|\lambda\bm{q}|}$ is the {\em intent}\/ of a firing expression
  $\bm{r}$ iff $\bm{q} \cong \bm{r}$. FIXME function $\eta\bm{r}$
  isomorphic to $\bm{r}$\qed
\end {definition}

Speaking informally, the set of firing expressions over ${\mathbb X}$
is ``free'' in the sense that there are no transformation rules
pre-imposed on the set of firing expressions (FIXME except
relabeling).  $\mathfrak{R}_{\mathbb X}$ is not of a ``specific
variety'', but instead, we expect that semantically induced
equivalences between firing expressions will lead to several algebraic
structures awaiting investigation --- transformation rules being a
consequence of particular semantic choices.  Next section clarifies
our understanding of {\em semantic equivalence}\/ of firing
expressions.

\subsection {Firing semantics}

\begin {definition}\label{def-semantics}
  A {\em firing semantics over}\/ ${\mathbb X}$ is any mapping
  $\bm{\varphi}$ associating a grill to every firing expression over
  ${\mathbb X}$, such that
  %
  \begin {enumerate}
  \item $\bm{r} \cong \bm{q} \implies \bm{\varphi}\bm{r} =
    \bm{\varphi}\bm{q}$,
    %
  \item $\bm{\varphi}\emptyset = \emptyset$, and
    %
  \item $\bm{\varphi}\{\sigma\} = [\sigma\rangle$ for all $\sigma \in
    \Sigma_{\mathbb X}$.
  \end {enumerate}
  %
  Two firing expressions $\bm{r}, \bm{q} \in \mathfrak{R}_{\mathbb X}$
  are $\bm{\varphi}$-equivalent iff $\bm{\varphi}\bm{r} =
  \bm{\varphi}\bm{q}$.\qed
\end {definition}

\begin {definition}\label{def-evaluation}
  Let ${\cal O}$ be a set of grill product operators and $\xi$ --- a
  function ${\mathbb Z}_{[1,z]} \rightarrow {\cal O}$ (called {\em
    substitution}\/), where $z = |\!\domain{\xi}|$ is some
  non-negative integer.  For any universe ${\mathbb X}$ we define the
  {\em $\xi$-evaluation}\/ as the mapping $\bm{\xi}$ that takes a
  firing expression $\bm{r} \in \mathfrak{R}_{\mathbb X}$ and gives a
  grill from $\bm{\Sigma}_{\mathbb X}$ such that
  %
  \begin {enumerate}
    \item if $\bm{r} \in \bm{\Sigma}_{\mathbb X}$, then
      $\bm{\xi}\bm{r} = \bm{r}$;
      %
    \item if $\bm{r} = (\bm{p}, \bm{q}, k)$ and $k \leq z$, then
      $\bm{\xi}\bm{r} = \bm{\xi}\bm{p} \bm{\circ} \bm{\xi}\bm{q}$,
      where $\bm{\circ} = \xi k$;
      %
    \item if $\bm{r} = (\bm{p}, \bm{q}, k)$ and $k > z$, then
      $\bm{\xi}\bm{r} = \bm{\xi}\bm{p} \cup \bm{\xi}\bm{q}$.
  \end {enumerate}

  FIXME relabeling

  The {\em $\xi$-semantics over}\/ ${\mathbb X}$ is the mapping of
  firing expressions to $\cup$\,-shells of grills obtained by
  $\xi$-evaluation.  The number $|\!\domain{\xi}|$ is called the {\em
    rank}\/ of $\xi$-semantics.\qed
\end {definition}

It is easy to check that, for any $\xi$ and any ${\mathbb X}$, the
$\xi$-semantics over ${\mathbb X}$ satisfies all three conditions
imposed on firing semantics in definition~\ref{def-semantics}.

\section {Final remarks}

FIXME Three guiding applications:
%
\begin {itemize}
\item parallel firing of multiple transitions with overlapping
  support: shouldn't this be solved through composition? note, that
  parallel firing of disjoint transitions is equivalent to
  interleaving (is it?);

\item composition: by exclusive choice (inclusive choice is xor of
  par and xor), parallel composition, sequential composition;

\item specifying weights.
\end {itemize}

FIXME sequential composition is implicit.

FIXME crn, sdf

\end {document}
