\documentclass [a4paper,12pt] {article}
\usepackage{fullpage}
\usepackage[utf8]{inputenc}
\usepackage{amsmath, amssymb, amsthm, mathabx, bm, mathtools, relsize, accents, cancel}
\usepackage{wasysym}
\usepackage{tikz, tabu}
\usepackage{stackengine}
\usepackage{enumitem}

\theoremstyle{definition}
\newtheorem{definition}{Definition}[section]
\newtheorem{lemma}{Lemma}[section]
\newtheorem{example}{Example}[section]
\newtheorem{remark}{Remark}[section]

\setcounter{equation}{0}
\renewcommand{\theequation}{\arabic{section}.\arabic{equation}}

\newcommand{\Forall}{}
\DeclareRobustCommand{\Forall}{\mathop{\mathlarger{\forall}}}
\newcommand{\Exists}{}
\DeclareRobustCommand{\Exists}{\mathop{\mathlarger{\exists}}}
\newcommand{\Qed}{}
\DeclareRobustCommand{\Qed}{\tag*{\qed}}
\newcommand{\inconflict}{}
\DeclareRobustCommand{\inconflict}{\mathbin{\natural}}
\newcommand{\exconflict}{}
\DeclareRobustCommand{\exconflict}{\mathbin{\sharp}}
\newcommand{\absorption}{}
\DeclareRobustCommand{\absorption}{\mathbin{\flat}}
\newcommand{\symdiff}{}
\DeclareRobustCommand{\symdiff}{\mathbin{\triangle}}

\newcommand{\undercone}{}
\DeclareRobustCommand{\undercone}{\mathop{\downarrow}}
\newcommand{\overcone}{}
\DeclareRobustCommand{\overcone}{\mathop{\uparrow}}

% FIXME superscript
\newcommand{\domain}{}
\DeclareRobustCommand{\domain}{\mathop{\textstyle\mathsmaller{\bf {Dom}}}}
\newcommand{\image}{}
\DeclareRobustCommand{\image}{\mathop{\textstyle\mathsmaller{\bf {Im}}}}
\newcommand{\carrier}{}
\DeclareRobustCommand{\carrier}{\mathop{\textstyle\mathsmaller{\bf {Car}}}}
\newcommand{\relative}{}
\DeclareRobustCommand{\relative}{\mathop{\textstyle\mathsmaller{\bf {Rel}}}}
\newcommand{\negative}{}
\DeclareRobustCommand{\negative}{\mathop{\textstyle\mathsmaller{\bf {Neg}}}}
\newcommand{\scope}{}
\DeclareRobustCommand{\scope}{\mathop{\textstyle\mathsmaller{\bf {Sco}}}}
\newcommand{\uniform}{}
\DeclareRobustCommand{\uniform}{\mathop{\textstyle\mathsmaller{\bf {Rep}}}}
\newcommand{\move}{}
\DeclareRobustCommand{\move}{\mathop{\textstyle\mathsmaller{\bf {Mov}}}}
\newcommand{\shift}{}
\DeclareRobustCommand{\shift}{\mathop{\textstyle\mathsmaller{\bf {Shi}}}}
\newcommand{\scaling}{}
\DeclareRobustCommand{\scaling}{\mathop{\textstyle\mathsmaller{\bf {Sca}}}}
\newcommand{\composition}{}
\DeclareRobustCommand{\composition}{\mathop{\textstyle\mathsmaller{\bf {Com}}}}

\newcommand{\id}{}
\DeclareRobustCommand{\id}{\mathop{\textstyle{\rm {id}}}}

\newcommand{\xbot}{}
\DeclareRobustCommand{\xbot}{\mathop{\textstyle\mathsmaller{\bm\bot}}}
\newcommand{\xtop}{}
\DeclareRobustCommand{\xtop}{\mathop{\textstyle\mathsmaller{\bm\top}}}

\DeclareMathAlphabet{\xmathbb}{U}{BOONDOX-ds}{m}{n}

\newcommand{\ThetaXV}{}
\DeclareRobustCommand{\ThetaXV}{\Theta_{{\mathbb X}\times{\mathbb V}}}

\newcommand{\uni}{}
\DeclareRobustCommand{\uni}{{\textstyle\mathlarger{\mathfrak{u}}}}

\newcommand{\preset}[1]{\prescript{\bullet}{}{\!#1}}
\newcommand{\postset}[1]{#1^{\bullet}}

\newcommand{\varset}[2]{\mathop {\textbf {\small\em #1}\,_{#2}}}

\makeatletter
\newcommand\c@rcle[2]{\mathbin{\ooalign{\hidewidth$#1#2$\hidewidth\crcr$#1\ocircle$}}}
\newcommand\di@mond[2]{\mathbin{\ooalign{\hidewidth$#1#2$\hidewidth\crcr$#1$\raisebox{-0.06em}{$\Diamond$}}}}
\newcommand\cr@ss[2]{\mathbin{\ooalign{\hidewidth$#1#2$\hidewidth\crcr$#1$\raisebox{0.13em}{$\times$}}}}
\newcommand\bcr@ss[2]{\mathbin{\ooalign{\hidewidth$#1#2$\hidewidth\crcr$#1$\raisebox{0.09em}{$\bm{\times}$}}}}
\newcommand\bc@rc[2]{\mathbin{\ooalign{\hidewidth$#1#2$\hidewidth\crcr$#1$\raisebox{0.27em}{$\mathsmaller{\bm{\circ}}$}}}}
\newcommand{\oeq}{\mathbin{\mathpalette\c@rcle{\raisebox{0.08em}{$\mathsmaller{=}$}}}}
\newcommand{\obullet}{\mathbin{\mathpalette\c@rcle{\raisebox{0.08em}{$\mathsmaller{\bullet}$}}}}
\newcommand{\nbullet}{\mathbin{\raisebox{0.08em}{$\mathsmaller{\bullet}$}}}
\newcommand{\wxp}{\mathbin{\mathpalette\c@rcle{\raisebox{0.035em}{$\mathsmaller{\triangledown}$}}}}
\newcommand{\nxp}{\mathbin{\raisebox{0.035em}{$\mathsmaller{\triangledown}$}}}
\newcommand{\wct}{\mathbin{\mathpalette\c@rcle{\raisebox{0.085em}{$\mathsmaller{\vartriangle}$}}}}
\newcommand{\nct}{\mathbin{\raisebox{0.085em}{$\mathsmaller{\vartriangle}$}}}
\newcommand{\wtimes}{\mathbin{\mathpalette\c@rcle{\raisebox{0.085em}{$\mathsmaller{\times}$}}}}
\newcommand{\ntimes}{\mathbin{\mathpalette\di@mond{\raisebox{0.12em}{$\mathsmaller{\times}$}}}}
\makeatother

\begin {document}
\title {Flux types, their runs\\and state space decomposition\\(a
  draft)} \author {} \date {} \maketitle

\section {Introduction}

Our primary motivation is to learn how to answer some interesting
questions concerning state space of any cause-effect structure of
reasonable size --- and to pay for the answer with an acceptable
computational effort.  A more distant goal is to relate algebraic
properties of \mbox {c-e} structures to compositional aspects of their
behavior.  In practical terms, we want the ability to predict
behavioral consequences of certain structural operations (take it with
a grain of salt, of course).

\section {Basic notions}

To set the stage, we need symbols $\mathbb {N}$, $\mathbb {N}_0$,
$\mathbb {Z}$, $\mathbb {B}$, $\xmathbb {0}$, $\xmathbb {1}$, $\xbot$,
and $\xtop$, where first three symbols represent, respectively,
positive, nonnegative, and all integers, last four represent some
non-integer constants, and $\mathbb {B} = \{ \xmathbb {0}, \xmathbb
{1} \}$.  We also use shorthands, $\mathbb {N}_{\,\omega} = \mathbb
{N}\cup\{\xtop\}$, and $\mathbb {Z}_{\,\omega} = \mathbb
{Z}\cup\{\xbot,\xtop\}$.

We need to extend standard integer arithmetics by declaring that
$\xbot$ is the absorbing element of multiplication and the neutral
element of addition, and conversely, $\xtop$ is the absorbing element
of addition and the neutral element of multiplication in $\mathbb
{Z}_{\,\omega}$.  In particular, $z \cdot \xbot = \xbot$, $z + \xtop =
\xtop$, and $z + \xbot = z \cdot \xtop = z$, for any $z \in \mathbb
     {Z}_{\,\omega}$.  The extended integer operations are lifted
     (pointwise) so that they apply to functions of the type $X
     \rightarrow \mathbb {Z}_{\,\omega}$, where $X$ is any fixed set.
     Similarly, the linear ordering of integers is extended, by
     declaring that any integer is greater than $\xbot$ and less than
     $\xtop$, and lifted to the partial order $f \leq g \iff
     \forall_{x \in X}\,fx \leq gx$, where $f, g\!: X \rightarrow
     \mathbb {Z}_{\,\omega}$.  A constant $z \in \mathbb
             {Z}_{\,\omega}$ appearing in an expression may be
             interpreted as the constant function $X \rightarrow \{ z
             \}$, whenever such interpretation is feasible.

\subsection {Flux type}

\begin {definition}\label {def:flux-type}
Take an arbitrary, nonempty finite or countably infinite set $\mathbf
{X}$ of {\em variables}\/.  Call $\mathbf {S} = \mathbb {Z}^\mathbf
{X}$ the set of {\em markings}\/, and $\mathbf {V} = 2^{\mathbf {S}}$
--- the set of {\em scopes}\/.

Assume also, that there is a nonempty finite or countably infinite set
$\mathbf {A}$ of {\em actions}\/.  Call $\mathbf {W} = \mathbb
{Z}_{\,\omega}^{\mathbf {A}{\times}\mathbb {B}{\times}\mathbf {X}}$
the set of {\em weightings}\/, and for $a \in \mathbf {A}$, $x \in
\mathbf {X}$, declare $[a^-] = (a, \xmathbb {0})$, $[a^+] = (a,
\xmathbb {1})$, $[{}^-x] = (\xmathbb {0}, x)$, $[{}^+x] = (\xmathbb
         {1}, x)$, $[a^-x] = (a, \xmathbb {0}, x)$, and $[a^+x] = (a,
         \xmathbb {1}, x)$.

Given sets $\mathbf {X}$ and $\mathbf {A}$ as above, a {\em step
  relation}\/ is any set $\mathbf {T} \subseteq \mathbf
{V}{\times}\mathbf {W}{\times}\mathbf {S}{\times}\mathbf
{A}{\times}\mathbf {S}$, such that $(v, w, s, a, s') \in \mathbf {T}
\,\Longrightarrow\, s,s' \in v$, a {\em flux type}\/ is any so defined
triple, $(\mathbf {X}, \mathbf {A}, \mathbf {T})$, and a {\em
  subtype}\/ of it is any triple $(\mathbf {X'}, \mathbf {A'}, \mathbf
{T'})$, such that $\mathbf {X'} \subseteq \mathbf {X}$, $\mathbf {A'}
\subseteq \mathbf {A}$ and $\mathbf {T'} \subseteq \mathbf {T}$.  A
flux type is
%
\begin {itemize}
\item {\em finite}\/ iff $\mathbf {X}$, $\mathbf {A}$ and $\mathbf
  {T}$ are all finite,
  %
\item {\em semideterministic}\/ iff $\mathbf {T}$ is the graph of some
  partial function $\mathbf {V}{\times}\mathbf {W}{\times}\mathbf
  {S}{\times}\mathbf {A} \rightarrow \mathbf {S}$,
  %
\item {\em deterministic}\/ iff $\mathbf {T}$ is the graph of some
  partial function $\mathbf {V}{\times}\mathbf {W}{\times}\mathbf
  {S} \rightarrow \mathbf {A}{\times}\mathbf {S}$.
\end {itemize}
%
If $t = (v, w, s, a, s') \in \mathbf {T}$ for some flux type $(\mathbf
{X}, \mathbf {A}, \mathbf {T})$, then $t$ is a {\em step}\/ in
$\mathbf {T}$, $s$ is the {\em input marking}\/, and $s'$ is the {\em
  output marking}\/ of $t$.  Finally, {\em enablement relations}\/ are
subsets of the set $\{ (v, w, s, a) \in \mathbf {V}{\times}\mathbf
{W}{\times}\mathbf {S}{\times}\mathbf {A} \;|\; s \in v \}$.\qed
\end {definition}

\begin {remark}
  Whenever we declare that symbols $\mathbf {X}$, $\mathbf {A}$, and
  $\mathbf {T}$ denote components of a flux type, we also implicitly
  reserve symbols $\mathbf {S}$, $\mathbf {U}$, $\mathbf {V}$, and
  $\mathbf {W}$ to be understood in accordance with definitions \ref
  {def:flux-type} and \ref {def:canonical-scope} --- and to be
  interpreted as if they were parameterized: $\mathbf {S}_{\mathbf
    {X}}$, $\mathbf {U}_{\mathbf {X}}$, $\mathbf {V}_{\mathbf {X}}$,
  and $\mathbf {W}_{\mathbf {X}, \mathbf {A}}$.
\end {remark}

\begin {remark}\label {flux-class}
   We sometimes call a collection of flux types their ``class'',
   without providing any formal definition --- roughly, a class
   contains all flux types possessing some property.  Most of our
   attention is focused on the ``canonical'' class (see definition
   \ref {def:canonical-consistency}).  A similar notion is a
   ``family'' --- a maximal subset of a given collection of flux
   types, such that all pairs of family members satisfy some property
   (for example, at least one member is homomorphic to the other).
\end {remark}

It is convenient to use custom names of some subsets of
$\mathbf {X}$:
%
\begin {align*}
\qquad\qquad\qquad\textit {pre-set}\/:&&\varset {pre}{w}a =&\; \{ x \in \mathbf {X} \;|\; w[a^-x] \neq \xbot \},\qquad\qquad\qquad\\
\qquad\qquad\qquad\textit {post-set}\/:&&\varset {post}{w}a =&\; \{ x \in \mathbf {X} \;|\; w[a^+x] \neq \xbot \},\qquad\qquad\qquad\\
\qquad\qquad\qquad\textit {residuals}\/:&& \varset {res}{w}a =&\; \{ x \in \mathbf {X} \;|\; w[a^-x] = w[a^+x] = \xbot \},\qquad\qquad\qquad\\
\qquad\qquad\qquad\textit {actuators}\/:&& \varset {act}{w}a =&\; \{ x \in \mathbf {X} \;|\; w[a^-x] = 0 \},\qquad\qquad\qquad\\
\qquad\qquad\qquad\textit {anihilators}\/:&& \varset {ani}{w}a =&\; \{ x \in \mathbf {X} \;|\; w[a^+x] = 0 \},\qquad\qquad\qquad\\
\qquad\qquad\qquad\textit {inhibitors}\/:&& \varset {inh}{w}a =&\; \{ x \in \mathbf {X} \;|\; w[a^-x] = \xtop \},\qquad\qquad\qquad\\
\qquad\qquad\qquad\textit {exhibitors}\/:&& \varset {exh}{w}a =&\; \{ x \in \mathbf {X} \;|\; w[a^+x] = \xtop \},\qquad\qquad\qquad
\end {align*}
%
where $a \in \mathbf {A}$ and $w \in \mathbf {W}$.  Customary
notation: $\preset{a} = \varset {pre}{w}a$, $\postset{a} = \varset
{post}{w}a$ and $\postset{\preset{a}} = \preset{a} \cup \postset{a}$,
may be applied whenever weighting $w$ is fixed.

\begin {definition}\label {def:consistency-and-instance}
  We say that an action, input marking, output marking, scope,
  weighting, or any combination of the five, is {\em consistent}\/
  with a given step relation, if it is contained in the corresponding
  projection of that relation.  A projection of arity 2 or greater is
  called, in its entirety, the {\em maximally consistent}\/ relation.

  In particular, given step relation $\mathbf {T}$, we define the {\em
    instantiation}\/ relation as the projection of $\mathbf {T}$ onto
  its first two coordinates,
%
\begin {equation*}
  \mathbf {Q}_{\mathbf {T}} = \{\, (v, w) \in \mathbf {V}{\times}\mathbf {W} \;|\;
  \Exists_{(s, a, s')\,\in\,v{\times}\mathbf {A}{\times}v}\,
  (v, w, s, a, s') \in \mathbf {T}\, \},
\end {equation*}
%
and similarly, the {\em enablement}\/ relation is defined as
%
\begin {equation*}
  \mathbf {E}_{\mathbf {T}} = \{\, (v, w, s, a) \in \mathbf {V}{\times}\mathbf {W}{\times}\mathbf {S}{\times}\mathbf {A}
  \;|\; \Exists_{s'\,\in\, v}\,(v, w, s, a, s') \in \mathbf {T}\, \}.
\end {equation*}

Any pair of scope and weighting consistent with step relation $\mathbf
{T}$ is called an {\em instantiation}\/ of $\mathbf {T}$.  A flux type
is called {\em instantiated}\/ iff its step relation has exactly one
instantiation.  By a {\em flux instance}\/ of some flux type we
understand any of its maximal instantiated subtypes.

For an instantiation $(v, w) \in \mathbf {Q}_{\mathbf {T}}$ we define
that its
%
\begin {itemize}
\item {\em effective scope}\/ is the set $\bm {l}_1v\cup\bm {r}_1v$
  (see definition \ref {def:lr-scopes}),
  %
\item {\em slicing operator}\/ $\bm {s}\!: \mathbf {A} \rightarrow
  2^{\mathbf {S}^2}$ is the mapping $a \mapsto \{\, (s, s') \in
  v{\times}v \;|\; (v, w, s, a, s') \in \mathbf {T}\, \}$, extended to
  $\mathbf {A} \supseteq A \mapsto \bigcup_{a \in A}\bm {s}a$,
%
\item {\em marking graph}\/ is the directed graph with vertex set $v$
  and edge set $\bm {s}\mathbf {A}$, and subgraphs induced by a single
  action are called {\em slices}\/.
\end {itemize}
%
An instantiation $(v, w) \in \mathbf {Q}_{\mathbf {T}}$ is
%
\begin {itemize}
\item {\em transitive}\/ iff for all $s, s', s'' \in v$,
  %
  \begin {equation*}
    \Exists_{a,a' \in \mathbf {A}}\,((v,w,s,a,s'), (v,w,s',a',s'')) \in \mathbf {T}^2
    \Longrightarrow \Exists_{a'' \in \mathbf {A}}\,(v,w,s,a'',s'') \in \mathbf {T},
  \end {equation*}
  %
\item {\em symmetric}\/ iff for all $s, s' \in v$,
  %
  \begin {equation*}
    \Exists_{a \in \mathbf {A}}\,(v,w,s,a,s') \in \mathbf {T}
    \iff \Exists_{a' \in \mathbf {A}}\,(v,w,s',a',s) \in \mathbf {T},
  \end {equation*}
  %
\item {\em reflexive}\/ iff $s \in v \Longrightarrow \Exists_{a \in
  \mathbf {A}}\,(v,w,s,a,s) \in \mathbf {T}$,
  %
\item {\em irreflexive}\/ iff $ \Forall_{s \in v, a \in \mathbf
  {A}}\,(v,w,s,a,s) \not\in \mathbf {T}$,
  %
\item {\em complete}\/ iff $v = \bm {l}_1v \cup \bm {r}_1v$,
  %
\item {\em total in}\/ $u \subseteq v$ iff $u \ni s \neq s' \in u
  \Longrightarrow \Exists_{a \in \mathbf {A}}\,(v, w, s, a, s') \in
  \mathbf {T} \,\lor\, (v, w, s', a, s) \in \mathbf {T}$,
  %
\item {\em effectively total}\/ iff it is total in $\bm {l}_1v \cup
  \bm {r}_1v$,
  %
\item {\em connected}\/ iff all its transitive closures are
  effectively total,
  %
\item {\em total}\/ iff it is total in $v$.
\end {itemize}
%
A flux type is transitive iff all instantiations of its step relation
are transitive --- and similarly for other properties shared by all
instantiations.\qed
\end {definition}

\begin {remark}
  If there is more than one action in $\mathbf {A}$, then any step
  relation over $\mathbf {A}$ has many transitive closures.  However,
  they are all FIXME family.  Therefore, in particular, all transitive
  closures are total in some scope iff any of them is.
\end {remark}

\subsection {Live and reachable scopes}

\begin {definition}\label {def:lr-scopes}
Take a flux type $(\mathbf {X}, \mathbf {A}, \mathbf {T})$, together
with sets $\mathbf {V}$ and $\mathbf {W}$ as in definition \ref
{def:flux-type}, and with parameters: $k \in \mathbb {N}_0$ and $q =
(v, w) \in \mathbf {Q}_{\mathbf {T}}$.  Scope transformation
operators, $\bm {l}$, $\bm {r}$, $\bm {L}$, $\bm {R}\!: \mathbf {V}
\rightarrow \mathbf {V}$, are mappings such that for every $u
\subseteq v$,
%
\begin {align*}
  &\;\bm {l}_{0,q} = \bm {r}_{0,q} = \bm {L}_{0,q} = \bm {R}_{0,q} = \id,\\
  %
  &\begin {rcases}
  \bm {l}_{k,q}u\!\!\!\!&=\, \{\, s \in u \;|\; \Exists_{a \in \mathbf {A},\, s' \in \bm {l}_{k-1,q}u}\,
  (v, w, s, a, s') \in \mathbf {T}\, \}\\
  %
  \bm {r}_{k,q}u\!\!\!\!&=\, \{\, s \in u \;|\; \Exists_{a \in \mathbf {A},\, s' \in \bm {r}_{k-1,q}u}\,
  (v, w, s', a, s) \in \mathbf {T}\, \}\\
  %
  \bm {L}_{k,q}u\!\!\!\!&=\, \{\, s \in v \;|\; \Exists_{a \in \mathbf {A},\, s' \in \bm {L}_{k-1,q}u}\,
  (v, w, s, a, s') \in \mathbf {T}\, \}\\
  %
  \bm {R}_{k,q}u\!\!\!\!&=\, \{\, s \in v \;|\; \Exists_{a \in \mathbf {A},\, s' \in \bm {R}_{k-1,q}u}\,
  (v, w, s', a, s) \in \mathbf {T}\, \}\quad
  \end {rcases}
  \textrm {for }k > 0,\\
  %
  &\;\bm {l}_q \;= \bigcap_{k>0}\bm {l}_{k,q},\quad\quad\bm {r}_q = \bigcap_{k>0}\bm {r}_{k,q},\\
  %
  &\;\bm {L}_q = \bigcup_{k>0}\bm {L}_{k,q},\quad\;\bm {R}_q = \bigcup_{k>0}\bm {R}_{k,q}.
\end {align*}
%
If $q$ is known, then we use shorthands: $\bm {l}_k = \bm {l}_{k,q}$,
$\bm {l} = \bm {l}_q$, and similarly for $\bm {r}$, $\bm {L}$ and $\bm
{R}$.

We call $\bm {l}_ku$ (resp.\ $\bm {r}_ku$) the $k$-{\em live}\/
(resp.\ $k$-{\em reachable}\/) {\em subscope}\/ of $u$, and $\bm
{L}_ku$ (resp.\ $\bm {R}_ku$) --- the $k$-{\em live prescope}\/
(resp.\ $k$-{\em reachable postscope}\/) of $u$.  Similarly, $\bm
{l}u$ is the live subscope, $\bm {r}u$ is the reachable subscope, $\bm
{L}u$ is the prescope, and $\bm {R}u$ is the postscope of $u$.  The
set $\bm {o}u = \bm {l}u \cap \bm {r}u$ is the {\em underscope}\/ of
$u$, and the set $\bm {O}u = u \cup \bm {L}u \cup \bm {R}u$ is the
{\em overscope}\/ of $u$.

Two scopes, $u, u' \subseteq v$ are $\bm {l}_k$-equivalent
(resp.\ $\bm {l}$-equivalent) iff $\bm {l}_ku = \bm {l}_ku'$
(resp.\ $\bm {l}u = \bm {l}u'$) --- and similarly for $\bm {r}$, $\bm
{o}$, $\bm {L}$, $\bm {R}$ and $\bm {O}$.\qed
\end {definition}

\begin {remark}
  The $k$-live subscope $\bm {l}_ku$ (resp.\ $k$-reachable subscope
  $\bm {r}_ku$) is the subset of $u$ consisting of all initial
  (resp.\ final) markings of a $k$-step evolution restricted to $u$.
  The $k$-live prescope $\bm {L}_ku$ (resp.\ $k$-reachable postscope
  $\bm {R}_ku$) is the subset of $v$ consisting of all initial
  (resp.\ final) markings of a $k$-step evolution ending at
  (resp.\ starting from) a marking in $u$.
\end {remark}

\begin {remark}
  FIXME modalities: eventualy dead scope is different from potentially
  dead scope
\end {remark}

Properties.

All scope transformation operators are non-decreasing:
$u \subseteq u' \Longrightarrow \bm {\tau}u \subseteq \bm {\tau}u'$,
where $\bm {\tau}$ is one of $\bm {l}$, $\bm {r}$, etc.  Subscope and
underscope operators are anti-extensive, overscope operator is
extensive.

FIXME Fixpoints

\subsection {Canonical consistency}

From now on, we abandon the general settings, and focus instead on the
class of ``canonical'' flux types (see remark \ref {flux-class}) and
their canonical instances --- specifically reflecting the behavior of
extended \mbox {c-e} structures in accordance with the original
formulation of their semantics.

\begin {definition}\label {def:canonical-scope}
  Given a flux type $(\mathbf {X}, \mathbf {A}, \mathbf {T})$, a {\em
    scope interval}\/ is any nonempty intersection of $\mathbf {S} =
  \mathbb {Z}^{\mathbf {X}}$ with a closed interval in $\mathbb
          {Z}_{\omega}^{\mathbf {X}}$ under $\leq$.  A scope interval
          $u \subseteq \mathbf {S}$ is {\em canonical}\/ iff it is
          nowhere negative,
  %
  \begin {equation*}
  \Exists_{s_0,\,s_1}\,
  s_0 = \mathsmaller{\bigwedge}u \in \mathbb {N}_0^\mathbf {X} \,\land\,
  s_1 = \mathsmaller{\bigvee}u \in \mathbb {N}_\omega^\mathbf {X} \,\land\,
  u = \{ s \in \mathbf {S} \;|\; s_0 \leq s \leq s_1 \},
  \end {equation*}
  %
  A scope $u \subseteq \mathbf {S}$ is {\em canonical}\/ iff it is a
  union of a nonempty finite collection of canonical scope intervals.
  The least upper bound of a canonical scope is called the {\em
    envelope}\/ of that scope.  The set of all canonical scopes is
  $\mathbf {U}$.\qed
\end {definition}

\begin {remark}
  The union of all canonical scopes is the set $\mathbb {N}_0^{\mathbf
    {X}}$, which forms a lattice under $\leq$, and the bounded lattice
  over $(\mathbb {N}_0\cup\{\xtop\})^{\mathbf {X}}$ is its completion.
  We write $\bigcup\mathbf {U}$ for the former, and $\bigsqcup\mathbf
  {U}$ for the latter.
\end {remark}

\begin {remark}
  For every scope in $\mathbf {U}$ a minimal decomposition into
  non-overlapping intervals exists, but is seldom unique.  FIXME
  clarify and extend.
\end {remark}

\begin {remark}
  The notion of canonical scope presented here is designed for
  behavioral analysis.  It is more general than necessary for static
  description, of course, because the actual set of all markings
  allowed for any cause-effect structure translates to a scope $u$,
  which has everywhere positive envelope (capacity assignment) and is
  always equal to a single interval: $u = [0, {\bigvee}u] \cap \mathbf
  {S}$, where ${\bigvee}u \geq 1$.
\end {remark}

In order to identify the class of canonical flux types, it is
sufficient to specify the quintuple canonical consistency predicate
(see definition \ref {def:consistency-and-instance}).  However, we
begin with lower arities, and incrementally develop the variadic
predicate $\bm {C}$.

\begin {definition}\label {def:canonical-consistency}
  A flux type $(\mathbf {X}, \mathbf {A}, \mathbf {T})$ is {\em
    canonical}\/ iff it is finite and semideterministic, and its step
  relation is defined by the {\em is-canonical}\/ predicate $\bm {C}$,
  such that for any $a \in \mathbf {A}$, $s, s' \in \mathbf {S}$, $u
  \in \mathbf {V}$, and $w \in \mathbf {W}$,
  %
\begin {align}\label{eq:is-canonical}\stepcounter{equation}
  \tag{\theequation.a}
  \bm {C}u \iff &u \in \mathbf {U},\\
  %
  \tag{\theequation.b}\label{eq:is-canonical-b}
  \bm {C}w \iff &\Forall_{x \in \mathbf {X}, a \in \mathbf {A}}\,w[a^-x] \cdot w[a^+x] \in \{\xbot, 0\}
  \;\land\; w[a^-x] + w[a^+x] \in \mathbb {N}_0 \cup \{ \xbot, \xtop \},\\
  %
  \tag{\theequation.c}\label{eq:is-canonical-c}
  \bm {C}(u, w) \iff &\bm {C}u \,\land\, \bm {C}w \,\land\, \Forall_{(a, b, x) \in \mathbf {A}{\times}\mathbb {B}{\times}\mathbf {X} }\;%
  w(a,b,x) > (\mathsmaller{\bigvee}u)x \Rightarrow w(a,b,x) = \xtop,\\
  %
  \tag{\theequation.d}\label{eq:is-canonical-d}
  \bm {C}(u, s) \iff &\bm {C}u \,\land\, s \in u,\\
  %
  \tag{\theequation.e}\label{eq:is-canonical-e}
  \bm {C}(u, w, s) \iff &\bm {C}(u, w) \,\land\, \bm {C}(u, s);
\end {align}
%
the quadruple $(u, w, s, a)$ is canonical iff $\bm {C}(u, w, s)$ and
the two {\em enablement conditions}\/,
%
\begin {align}\label {eq:enablement-conditions}
  \tag{\theequation.a}
  \Forall_{x \in \mathbf {X}}\;&%
  \begin {cases}
    sx = 0, &\textrm { if } w[a^-x] = \xtop\\
    \max(w[a^-x],1) \leq sx, &\textrm { if } w[a^-x] \in \mathbb {N}_0,
  \end {cases}\\
  %
  \intertext{and}
  %
  \tag{\theequation.b}
  \Forall_{x \in \mathbf {X}}\;&%
  \begin {cases}
    sx > 0, &\textrm { if } w[a^+x] = \xtop\\
    w[a^+x] \leq (\bigvee u)x - sx, &\textrm { otherwise},
  \end {cases}
\end {align}
%
hold --- if this is the case, we can say that action $a$ is {\em
  enabled}\/ in marking $s$ for a given $(u, w)$; finally, the
quintuple $(u, w, s, a, s')$ is canonical iff $\bm {C}(u, w, s, a)$,
$\bm {C}(u, s')$, and the {\em step condition}\/,
%
\begin {equation}\label {eq:step-condition}
\Forall_{x \in \mathbf {X}}\; s'x =
\begin {cases}
sx,& \text {if } w[a^-x] + w[a^+x] = \xtop\\
sx - w[a^-x] + w[a^+x],& \text {otherwise.}
\end {cases}
\end {equation}
%
is satisfied.\qed
\end {definition}

\begin {remark}
  By {\em canonical consistency}\/ we understand consistency with a
  step relation of some canonical flux type.  In particular, a pair
  $(u, w)$ of scope and weighting is canonically consistent iff there
  is a canonical flux type $(\mathbf {X}, \mathbf {A}, \mathbf {T})$,
  such that $(u, w, s, a, s') \in \mathbf {T}$ for some $(s, a, s')
  \in \mathbf {S}_{\mathbf {X}}{\times}\mathbf {A}{\times}\mathbf
      {S}_{\mathbf {X}}$ (see definition \ref
      {def:consistency-and-instance}).  Any such pair is called a {\em
        canonical instantiation}\/.  The symbol $\mathbf {Q}$ (without
      a subscript) denotes the set of all canonical instantiations,
      i.e.\ the union of $\mathbf {Q}_{\mathbf {T}}$ over all
      canonical $\mathbf {T}$.
\end {remark}

\begin {remark}
  Note, that $(u, w) \in \mathbf {Q}$ doesn't follow from $\bm {C}(u,
  w)$.  Take, for example, $|\mathbf {X}| = 2$, $\mathbf {A} = \{ a
  \}$, $u = \{ (0, 0), (2, 2) \}$, $w[a^-] = (\xtop, \xbot)$, and
  $w[a^+] = (\xbot, 1)$.  FIXME what are the sufficient conditions?
\end {remark}

In line with definition \ref {def:flux-type}, the quadruple and
quintuple forms of predicate $\bm {C}$ determine the canonical
enablement and step relations.  By sticking to a single canonical
scope and weighting we may reduce these relations to binary and
ternary, and apply special notation for improved readability: infix
relational operator $\searrow$ denotes binary enablement relation,
i.e.\ $\searrow = \{ (s, a) \in \mathbf {S}{\times}\mathbf {A} \;|\;
\bm {C}(u, w, s, a) \}$, and similarly, the reduced form of step
relation is $s\overset{a}{\rightarrow}s' \iff \bm {C}(u, w, s, a,
s')$.

\end {document}
