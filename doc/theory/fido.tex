\documentclass [a4paper,12pt] {article}
\usepackage{fullpage}
\usepackage[utf8]{inputenc}
\usepackage{amsmath, amssymb, amsthm, mathabx, bm, mathtools, relsize, accents, cancel}
\usepackage{wasysym}
\usepackage{tikz, tabu}
\usepackage{stackengine}
\usepackage{enumitem}

\theoremstyle{definition}
\newtheorem{definition}{Definition}[section]
\newtheorem{lemma}{Lemma}[section]

\setcounter{equation}{0}
\renewcommand{\theequation}{\arabic{section}.\arabic{equation}}

\newcommand{\Forall}{}
\DeclareRobustCommand{\Forall}{\mathop{\mathlarger{\forall}}}
\newcommand{\Exists}{}
\DeclareRobustCommand{\Exists}{\mathop{\mathlarger{\exists}}}
\newcommand{\Qed}{}
\DeclareRobustCommand{\Qed}{\tag*{\qed}}
\newcommand{\inconflict}{}
\DeclareRobustCommand{\inconflict}{\mathbin{\natural}}
\newcommand{\exconflict}{}
\DeclareRobustCommand{\exconflict}{\mathbin{\sharp}}
\newcommand{\absorption}{}
\DeclareRobustCommand{\absorption}{\mathbin{\flat}}
\newcommand{\symdiff}{}
\DeclareRobustCommand{\symdiff}{\mathbin{\triangle}}

\newcommand{\undercone}{}
\DeclareRobustCommand{\undercone}{\mathop{\downarrow}}
\newcommand{\overcone}{}
\DeclareRobustCommand{\overcone}{\mathop{\uparrow}}

% FIXME superscript
\newcommand{\domain}{}
\DeclareRobustCommand{\domain}{\mathop{\textstyle\mathsmaller{\bf {Dom}}}}
\newcommand{\image}{}
\DeclareRobustCommand{\image}{\mathop{\textstyle\mathsmaller{\bf {Im}}}}
\newcommand{\carrier}{}
\DeclareRobustCommand{\carrier}{\mathop{\textstyle\mathsmaller{\bf {Car}}}}
\newcommand{\relative}{}
\DeclareRobustCommand{\relative}{\mathop{\textstyle\mathsmaller{\bf {Rel}}}}
\newcommand{\negative}{}
\DeclareRobustCommand{\negative}{\mathop{\textstyle\mathsmaller{\bf {Neg}}}}
\newcommand{\scope}{}
\DeclareRobustCommand{\scope}{\mathop{\textstyle\mathsmaller{\bf {Sco}}}}
\newcommand{\uniform}{}
\DeclareRobustCommand{\uniform}{\mathop{\textstyle\mathsmaller{\bf {Rep}}}}
\newcommand{\move}{}
\DeclareRobustCommand{\move}{\mathop{\textstyle\mathsmaller{\bf {Mov}}}}
\newcommand{\shift}{}
\DeclareRobustCommand{\shift}{\mathop{\textstyle\mathsmaller{\bf {Shi}}}}
\newcommand{\scaling}{}
\DeclareRobustCommand{\scaling}{\mathop{\textstyle\mathsmaller{\bf {Sca}}}}
\newcommand{\composition}{}
\DeclareRobustCommand{\composition}{\mathop{\textstyle\mathsmaller{\bf {Com}}}}

\newcommand{\id}{}
\DeclareRobustCommand{\id}{\mathop{\textstyle{\rm {id}}}}

\newcommand{\xbot}{}
\DeclareRobustCommand{\xbot}{\mathop{\textstyle\mathsmaller{\bm\bot}}}
\newcommand{\xtop}{}
\DeclareRobustCommand{\xtop}{\mathop{\textstyle\mathsmaller{\bm\top}}}

\DeclareMathAlphabet{\xmathbb}{U}{BOONDOX-ds}{m}{n}

\newcommand{\ThetaXV}{}
\DeclareRobustCommand{\ThetaXV}{\Theta_{{\mathbb X}\times{\mathbb V}}}

\newcommand{\uni}{}
\DeclareRobustCommand{\uni}{{\textstyle\mathlarger{\mathfrak{u}}}}

\newcommand{\preset}[1]{\prescript{\bullet}{}{\!#1}}
\newcommand{\postset}[1]{#1^{\bullet}}

\newcommand{\varset}[2]{\mathop {\textbf {\small\em #1}\,_{#2}}}

\makeatletter
\newcommand\c@rcle[2]{\mathbin{\ooalign{\hidewidth$#1#2$\hidewidth\crcr$#1\ocircle$}}}
\newcommand\di@mond[2]{\mathbin{\ooalign{\hidewidth$#1#2$\hidewidth\crcr$#1$\raisebox{-0.06em}{$\Diamond$}}}}
\newcommand\cr@ss[2]{\mathbin{\ooalign{\hidewidth$#1#2$\hidewidth\crcr$#1$\raisebox{0.13em}{$\times$}}}}
\newcommand\bcr@ss[2]{\mathbin{\ooalign{\hidewidth$#1#2$\hidewidth\crcr$#1$\raisebox{0.09em}{$\bm{\times}$}}}}
\newcommand\bc@rc[2]{\mathbin{\ooalign{\hidewidth$#1#2$\hidewidth\crcr$#1$\raisebox{0.27em}{$\mathsmaller{\bm{\circ}}$}}}}
\newcommand{\oeq}{\mathbin{\mathpalette\c@rcle{\raisebox{0.08em}{$\mathsmaller{=}$}}}}
\newcommand{\obullet}{\mathbin{\mathpalette\c@rcle{\raisebox{0.08em}{$\mathsmaller{\bullet}$}}}}
\newcommand{\nbullet}{\mathbin{\raisebox{0.08em}{$\mathsmaller{\bullet}$}}}
\newcommand{\wxp}{\mathbin{\mathpalette\c@rcle{\raisebox{0.035em}{$\mathsmaller{\triangledown}$}}}}
\newcommand{\nxp}{\mathbin{\raisebox{0.035em}{$\mathsmaller{\triangledown}$}}}
\newcommand{\wct}{\mathbin{\mathpalette\c@rcle{\raisebox{0.085em}{$\mathsmaller{\vartriangle}$}}}}
\newcommand{\nct}{\mathbin{\raisebox{0.085em}{$\mathsmaller{\vartriangle}$}}}
\newcommand{\wtimes}{\mathbin{\mathpalette\c@rcle{\raisebox{0.085em}{$\mathsmaller{\times}$}}}}
\newcommand{\ntimes}{\mathbin{\mathpalette\di@mond{\raisebox{0.12em}{$\mathsmaller{\times}$}}}}
\makeatother

\begin {document}
\title {Firing domain sequences\\(a draft)}
\author {}
\date {}
\maketitle

\section {Introduction}

To set the stage we need symbols $\mathbb {N}$, $\mathbb {N}_0$,
$\mathbb {B}$, $\xmathbb {0}$, $\xmathbb {1}$, $\xbot$, and $\xtop$,
where first two symbols represent, respectively, positive and
nonnegative integers, last four represent some non-integer constants,
and $\mathbb {B} = \{ \xmathbb {0}, \xmathbb {1} \}$.  Then, we
introduce an arbitrary, finite or countably infinite set $\mathbf {X}$
of (``unsigned integer'') {\em variables}\/.  We call $\mathbf {U} =
(\mathbb {N}\cup\{\xtop\})^\mathbf {X}$ --- the set of {\em
  envelopes}\/ (upper bounds), and $\mathbf {S} = \mathbb
{N}_0^\mathbf {X}$ --- the set of {\em markings}\/.

We also assume that there is a finite set $\mathbf {A}$ of {\em
  actions}\/ (aka {\em aliens}\/), call $\mathbf {W} = (\mathbb
{N}_0\cup\{\xbot, \xtop\})^{\mathbf {A} \times \mathbb {B} \times
  \mathbf {X}}$ --- the set of {\em weightings}\/, and for $a \in
\mathbf {A}$, $x \in \mathbf {X}$ write $a^- = (a, \xmathbb {0})$,
$a^+ = (a, \xmathbb {1})$, $[a^-x] = (a, \xmathbb {0}, x)$, and
$[a^+x] = (a, \xmathbb {1}, x)$.  It is sometimes convenient to use
custom names of some subsets of $\mathbf {X}$:
%
\begin {align*}
\qquad\qquad\qquad\textit {pre-set}\/:&&\varset {pre}{w}a =&\; \{ x \in \mathbf {X} \;|\; w[a^-x] \neq \xbot \},\qquad\qquad\qquad\\
\qquad\qquad\qquad\textit {post-set}\/:&&\varset {post}{w}a =&\; \{ x \in \mathbf {X} \;|\; w[a^+x] \neq \xbot \},\qquad\qquad\qquad\\
\qquad\qquad\qquad\textit {residuals}\/:&& \varset {res}{w}a =&\; \{ x \in \mathbf {X} \;|\; w[a^-x] = w[a^+x] = \xbot \},\qquad\qquad\qquad\\
\qquad\qquad\qquad\textit {actuators}\/:&& \varset {act}{w}a =&\; \{ x \in \mathbf {X} \;|\; w[a^-x] = 0 \},\qquad\qquad\qquad\\
\qquad\qquad\qquad\textit {anihilators}\/:&& \varset {ani}{w}a =&\; \{ x \in \mathbf {X} \;|\; w[a^+x] = 0 \},\qquad\qquad\qquad\\
\qquad\qquad\qquad\textit {inhibitors}\/:&& \varset {inh}{w}a =&\; \{ x \in \mathbf {X} \;|\; w[a^-x] = \xtop \},\qquad\qquad\qquad\\
\qquad\qquad\qquad\textit {exhibitors}\/:&& \varset {exh}{w}a =&\; \{ x \in \mathbf {X} \;|\; w[a^+x] = \xtop \},\qquad\qquad\qquad
\end {align*}
%
where $a \in \mathbf {A}$ and $w \in \mathbf {W}$.  Whenever weighting
is fixed, the customary notation may be applied: $\preset{a}$ for a
pre-set and $\postset{a}$ for a post-set of action $a$.

The usual integer arithmetics is extended by declaring that $\xbot$ is
the absorbing element of multiplication and the neutral element of
addition, and conversely, $\xtop$ is the absorbing element of addition
and the neutral element of multiplication.  In particular, $n \cdot
\xbot = \xbot$, $n + \xtop = \xtop$, and $n + \xbot = n \cdot \xtop =
n$, for any $n \in \mathbb {N}_0$.  The extended integer operations
are lifted (pointwise) so that they apply to functions $\mathbf {X}
\rightarrow \mathbb {N}_0 \cup \{\xbot, \xtop\}$.  Similarly, the
linear ordering of integers is extended, by declaring that any integer
is greater than $\xbot$ and less than $\xtop$, and lifted to the
partial order $f \leq g \iff \forall_{x \in \mathbf {X}}\,fx \leq gx$,
where $f, g\!: \mathbf {X} \rightarrow \mathbb {N}_0 \cup \{\xbot,
\xtop\}$.

\subsection {Consistency}

Now we can introduce the notion of {\em consistency}\/ represented by
the predicate $\bm {C}$.  For any $s \in \mathbf {S}$, $u \in \mathbf
{U}$, and $w \in \mathbf {W}$,
  %
\begin {align}\label{eq:static-consistency}\stepcounter{equation}
  \tag{\theequation.a}
  \bm {C}(s, u) \iff &s \leq u,\\
  %
  \tag{\theequation.b}
  \bm {C}w \iff &\Forall_{x \in \mathbf {X}, a \in \mathbf {A}}\,w[a^-x] \cdot w[a^+x] \in \{\xbot, 0\},\\
  %
  \tag{\theequation.c}
  \bm {C}(u, w) \iff &\bm {C}w \,\land\, \Forall_{(a, b, x) \in \mathbf {A} \times \mathbb {B} \times \mathbf {X} }\;%
  w(a,b,x) > ux \Rightarrow w(a,b,x) = \xtop,\\
  %
  \tag{\theequation.d}
  \bm {C}(s, u, w) \iff &\bm {C}(s, u) \,\land\, \bm {C}(u, w).
\end {align}

The definition of consistency is developed further to cover more
complex objects.  Given an action $a \in \mathbf {A}$, the quadruple
$(a, s, u, w)$ is consistent iff $\bm {C}(s, u, w)$ and the {\em
  enablement condition}\/,
%
\begin {align}\label {eq:enablement-condition}
  \Forall_{x \in \mathbf {X}}\;&%
  \begin {cases}
    sx = 0, &\textrm { if } w[a^-x] = \xtop\\
    \max(w[a^-x],1) \leq sx, &\textrm { otherwise}
  \end {cases}
  %
  \land\;&%
  \begin {cases}
    sx > 0, &\textrm { if } w[a^+x] = \xtop\\
    w[a^+x] \leq ux - sx, &\textrm { otherwise},
  \end {cases}
\end {align}
%
holds (alternatively, we say that $a$ is {\em enabled}\/ in $s$ given
$u$ and $w$).  Moreover, given $a \in \mathbf {A}$ and $s' \in \mathbf
{S}$, the quintuple $(a, s, s', u, w)$ is consistent iff $\bm {C}(a,
s, u, w)$ and the {\em firing condition}\/,
%
\begin {equation}\label {eq:firing-condition}
\Forall_{x \in \mathbf {X}}\; s'x =
\begin {cases}
sx,& \text {if } w[a^-x] + w[a^+x] = \xtop\\
sx - w[a^-x] + w[a^+x],& \text {otherwise.}
\end {cases}
\end {equation}
%
is satisfied.  In what follows, the relation defined by the quadruple
consistency predicate is called {\em enablement consistency}\/, while
the quintuple variant specifies {\em transition consistency}\/
relation.  Fixing $u$ and $w$ reduces these relations to binary and
ternary --- in such cases special notation improves readability:
$s\searrow a$ denotes binary enablement consistency, i.e.\ $\searrow =
\{ (s, a) \in \mathbf {S} \times \mathbf {A} \;|\; \bm {C}(a, s, u, w)
\}$, and similarly, $s\overset{a}{\rightarrow}s'$ denotes ternary
transition consistency.

Clearly, the description in terms of relations between individual
actions and markings doesn't capture behavior, therefore the next step
is to explore relations between sequences of actions and markings.

\subsection {Firing sequence}

In all relevant cases it is possible to modify statements about
consistency by replacing individual elements of $\mathbf {A}$ and
$\mathbf {S}$ with elements of free monoids $\mathbf {A}^{\!\ast}$ and
$\mathbf {S}^\ast$.  We adopt the symbol $.$ as the concatenation
operator (usually omitted), define $\mathbf {A}^{\!k} = \{ \alpha \in
\mathbf {A}^{\!\ast} \;|\; |\alpha| = k \}$ (same for $\mathbf
        {S}^k$), where $|\alpha|$ denotes the length of word $\alpha$,
        and use parentheses to distinguish single-letter words from
        letters themselves --- thus, in particular, $\mathbf {A}^{\!1}
        = \{ (a) \;|\; a \in \mathbf {A} \}$.

Let us start from informal definition of consistency for the quadruple
case.

\begin {quote}
    Given $u \in \mathbf {U}$ and $w \in \mathbf {W}$, two words, one
    from $\mathbf {A}^{\!\ast}$, another from $\mathbf {S}^\ast$, are
    consistent iff enablement and firing conditions are satisfied for
    all corresponding actions and markings.
\end {quote}

\noindent
We say that an action sequence with a corresponding marking sequence
jointly form a {\em firing sequence}\/ --- if they are constistent for
a given $u$ and $w$.

There may be a partial correspondence between action and marking
sequences.  If $|\alpha| + 1 < |\sigma|$, then action sequence
$\alpha$ has to be consistent with the corresponding prefix of the
given marking sequence $\sigma$, while the remaining part of $\sigma$
is expected to be matched by at least one action sequence $\alpha'$ of
proper length, such that the enablement and firing conditions are
satisfied throughout --- and conversely for $|\alpha| + 1 > |\sigma|$.
The {\em length}\/ of a firing sequence $(\alpha, \sigma)$ is the
number $\max (|\alpha|, |\sigma| - 1)$.

Formally, we introduce the sequential consistency predicate $\bm
{C}_k$ parameterized by firing sequence length $k \in \mathbb {N}_0$,
and define the {\em run consistency}\/ relation.

\begin {definition}\label{def:k-consistency}
  Assuming that $a \in \mathbf {A}$, $s, s' \in \mathbf {S}$, $\alpha,
  \alpha' \in \mathbf {A}^{\!\ast}$, $\sigma, \sigma' \in \mathbf
          {S}^\ast$, $u \in \mathbf {U}$, and $w \in \mathbf {W}$ are
          arbitrary, $\lambda_{\mathbf {A}} \in \mathbf {A}^{\!\ast}$,
          $\lambda_{\mathbf {S}} \in \mathbf {S}^{\ast}$ are empty
          words, and $k \in \mathbb {N}_0$,

\begin {enumerate}[leftmargin=16pt]
  \item a quadruple $(\alpha, \sigma, u, w)$ is $k$-consistent,
    symbolically $\bm {C}_k(\alpha, \sigma, u, w)$, iff one of the
    following is true:
%
\begin {enumerate}[leftmargin=16pt]
\item[a.] $\displaystyle\alpha = \lambda_{\mathbf {A}} \,\land\, \sigma = (s) \;\land\; \bm {C}(s, u, w)$,
  %
\item[b.] $\displaystyle|\alpha| = k = |\sigma| - 1 \;\land\;
  \alpha = (a)\alpha' \,\land\, \sigma = (ss')\sigma' \;\land\; \bm {C}(a, s, s', u, w) \,\land\,\\
  \,\land\, \bm {C}_{k-1}(\alpha', (s')\sigma', u, w)$,
  %
\item[c.] $\displaystyle|\alpha| < k = |\sigma| - 1 \;\land\;
  \Exists_{\alpha' \in \mathbf {A}^{\!|\sigma| - |\alpha| - 1}}\,\bm {C}_k(\alpha\alpha', \sigma, u, w)$,
  %
\item[d.] $\displaystyle|\alpha| = k > |\sigma| - 1 \;\land\;
    \Exists_{\sigma' \in \mathbf {S}^{|\alpha| + 1 - |\sigma|}}\,\bm {C}_k(\alpha, \sigma\sigma', u, w)$,
\end {enumerate}

\item $k$-consistency of triples is given by
%
\begin {enumerate}[leftmargin=16pt]
\item[a.] $\displaystyle\bm {C}_k(\alpha, u, w) \iff \Exists_{\sigma \in \mathbf {S}^\ast}\,\bm {C}_k(\alpha, \sigma, u, w)$,
  %
\item[b.] $\displaystyle\bm {C}_k(\sigma, u, w) \iff \Exists_{\alpha \in \mathbf {A}^{\!\ast}}\,\bm {C}_k(\alpha, \sigma, u, w)$,
\end {enumerate}

\item {\em run consistency}\/ relation is the set of quadruples,
%
\begin {equation*}
  \{ (\alpha, \sigma, u, w) \in \mathbf {A}^{\!\ast} \times \mathbf {S}^{\ast} \times \mathbf {U} \times \mathbf {W}
  \;|\; \bm {C}_{\max (|\alpha|, |\sigma| - 1)}(\alpha, \sigma, u, w) \}.\qed
\end {equation*}
\end {enumerate}
\end {definition}

\noindent
If $u$ and $w$ are fixed, then the preferred simplified form of run
consistency is a binary relation $\searrow \subseteq \mathbf
{S}^{\ast} \times \mathbf {A}^{\!\ast}$.

\end {document}
